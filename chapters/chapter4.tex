%%% 讨论

\section{研究成果总结}

本研究基于 Julia 编程语言成功设计并实现了一个高性能的单细胞数据分析平台 —— Juscan.jl。该平台以 Muon.jl 所提供的 AnnData 数据结构为核心,复刻并优化了 Scanpy 在 Python 生态中的分析流程,完成了从数据质量控制、标准化、高变基因筛选、降维、聚类到可视化的完整管线,构建出一个功能完备、接口友好、计算高效的分析框架。

更重要的是,Juscan.jl 采用模块化设计,具有良好的可扩展性和可维护性,为后续的算法开发和功能拓展提供了坚实的基础。

\section{未来工作展望}


尽管 Juscan.jl 已实现了基础的单细胞分析功能,但作为一个初步探索性的项目,仍存在许多可提升的方向,具体包括以下几个方面:

\begin{enumerate}
\item 多模态数据支持与整合: 当前版本主要面向 scRNA-seq 数据,未来可进一步支持 ATAC-seq、CITE-seq 以及空间转录组等多组学数据,并结合 Canonical Correlation Analysis、WNN等多模态聚类方法,实现更全面的细胞表征。
\item 更多聚类与轨迹推断算法的集成: 目前聚类模块仅支持 Louvain 与 k-medoids,后续可引入 Leiden 算法、基于高斯混合模型(GMM)的聚类方法,甚至构建基于图卷积神经网络(GCN)的聚类与分类器。同时,可引入 pseudotime 轨迹推断模块,满足发育过程建模的需求。
\item 计算性能的进一步优化: 尽管 Julia 语言本身具备极高的运行性能,但本项目中的部分性能瓶颈主要源于代码层面的实现尚不够成熟。例如,部分函数存在数据复制冗余、内存访问不充分优化、缺乏懒计算等问题,影响了整体的执行效率与稳定性。未来可以通过进一步剖析关键函数的瓶颈(如使用 @profile、@btime 等工具)、重构核心逻辑、合理安排内存布局、并结合并行/多线程机制等方式,显著提升 Juscan.jl 的运行效率与工程质量。同时,也可结合 PackageCompiler.jl 生成系统映像,以减少启动与冷启动时的性能损耗,从而带来更流畅的用户体验。
\item 用户交互体验与可视化增强: 当前可视化模块虽已支持 UMAP、提琴图等常规图形,但仍缺乏交互式浏览器支持。可借助 Makie.jl + Genie.jl 或 Pluto.jl 构建 Web 可视化平台,让用户通过图形界面进行数据探索。
\end{enumerate}
