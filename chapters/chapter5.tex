本研究以秀丽隐杆线虫为模式生物,深入探讨了组织蛋白酶家族在体内屏障衰老过程  中的作用与机制。通过该实验,我们不仅验证了既往关于 CPR-6 组织蛋白酶的研究结论, 还扩展了组织蛋白酶家族在衰老过程中的功能谱系。我们成功鉴定了 40 种线虫体内组织  蛋白酶同源物,并通过 smurf assy 实验和观察HMR-1蛋白的分布筛选出了能够显著破坏肠 道和表皮屏障完整性的关键基因。这些基因的表达水平与线虫衰老过程中的屏障功能衰退  呈现显著相关性,证实了组织蛋白酶家族通过降解细胞间连接蛋白破坏组织屏障的分子机  制。这一发现为理解衰老相关病理变化提供了新的视角。

但是在进行 RNAi 实验的过程中,我们发现一些基因被敲低之后,线虫的存活率显著 下降,这可能是由于这几种基因是线虫中较为重要的基因。但至于为什么如此重要的基因 依旧会在线虫衰老的过程中破坏线虫的组织屏障,它们在线虫中的具体作用还有待研究。

本次实验为后续工作提供了多个值得深入探索的方向。比如,我们可以进一步探索这  些蛋白酶与其他衰老相关信号通路的交互作用, 这可能揭示更复杂的衰老调控网络。此外, 鉴于线虫 HMR-1 蛋白与人类连接蛋白的同源性,后续研究我们可将发现的靶点转化到哺  乳动物模型中去,为临床应用提供直接依据。

在应用层面,本研究的成果可能成为延缓组织屏障功能衰退、改善老年相关疾病的创 新疗法。同时,结合基因编辑技术和 RNA 干扰疗法,我们有望实现对衰老过程的精准干 预。

最后,实验的成功离不开实验团队成员的协作与创新。我们期待未来能将更多的基础 研究成果转化为临床应用。随着未来研究的深入,组织蛋白酶家族在衰老医学中的潜力将 被进一步挖掘,为人类健康事业开辟新的篇章。
