在生物体中,细胞间的连接包括紧密连接、黏附连接和间隙连接, 这些连接共同构成了 生物组织屏障的基础,确保了细胞间的正常连接交流和组织内部环境的稳定\upcite{noauthor__nodate}。然而,随着 机体的衰老, 细胞连接被逐渐破坏,出现组织屏障功能障碍以及生理功能紊乱的现象\upcite{bloom_mechanisms_2023} 。 在秀丽隐杆线虫这一经典模式生物中,表皮屏障作为抵御外界环境的第一道防线,其结构 完整性主要依赖于细胞间连接蛋白 HMR-1 。HMR-1 蛋白在维持表皮组织的紧密连接和屏 障功能方面发挥着关键作用\upcite{klompstra_instructive_2015}。前期研究显示, 随着线虫的衰老,表皮屏障中的 HMR-1 蛋 白逐渐降解。更深入的分子机制研究表明, 组织蛋白酶 B 家族在线虫中的同源物 CPR-6 在 此过程中起到降解 HMR-1 的作用,破坏线虫屏障的完整性。同时, 在线虫中,组织蛋白酶 家族成员众多且功能重叠度高, 这不由得引起我们的猜想,组织蛋白酶家族的其他成员是 否在衰老过程中参与降解胞间连接蛋白 HMR-1,并且最终导致机体出现严重的生理功能  障碍。基于上述研究背景, 本研究以秀丽隐杆线虫为模式生物, 探究组织蛋白酶家族在线  虫中的其他同源物是否具有与 CPR-6 类似的调控功能,我们利用 RNA 干扰技术特异性敲  低这些基因,并结合线虫组织屏障检查的相关实验, 筛选出了一些和 CPR-6 具有相同作用 的同源基因。通过这一研究,我们首次系统阐明了组织蛋白酶家族调控组织屏障稳态的分 子机制。本研究的结果不仅完善了衰老相关屏障功能障碍的理论框架,更为开发靶向组织 蛋白酶家族的抗衰老策略提供了重要理论依据。通过本研究,我们有望找到新的干预靶点 , 延缓组织屏障功能的衰退,延长生物体的健康寿命,具有重要的科学意义。
