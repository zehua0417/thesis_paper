% !TEX TS-program = xelatex
% !TEX encoding = UTF-8 Unicode

% \documentclass[AutoFakeBold]{LZUThesis}
\documentclass[AutoFakeBold]{LZUThesis}

\usepackage{wasysym}
\usepackage{enumitem}
\usepackage[most]{tcolorbox}
\usepackage{multirow}
\usepackage{inputenc}
\usepackage{tikz}
\usepackage{bbding}
\usetikzlibrary{arrows.meta, decorations.markings}
\usepackage{hyperref}
\usepackage[numbers,sort&compress]{natbib}
\usepackage{pdfpages}
% \newcommand{\upcite}[1]{\textsuperscript{\textsuperscript{\cite{#1}}}}
\allowdisplaybreaks[4]
\usepackage{pdfpages}
\usepackage[many]{tcolorbox}
\usepackage{setspace}
% \setmonofont{MapleMono-NF-CN-Medium}
\setmonofont{MapleMono-NF-CN-Regular}

% \newtcolorbox[auto counter, number within=section, list inside=listoftables]{fancybox}[2][]{
\newcommand{\supercite}[1]{\textsuperscript{\textsuperscript{\cite{#1}}}}
\newtcolorbox[use counter=table]{fancybox}[2][]{
    title=Code~\thetcbcounter: #2,
    breakable, left=1cm,
    label={code:#2},
    % list entry=code~\thetcbcounter: #2
    % colback=yellow!10!white, colframe=red!50!black
}








\usepackage[dvipsnames]{xcolor}
\usepackage{tikz}
\usetikzlibrary{backgrounds}
\usetikzlibrary{arrows,shapes}
\usetikzlibrary{tikzmark}
\usetikzlibrary{calc}

\usepackage{amsmath}
\usepackage{amsthm}
\usepackage{amssymb}
\usepackage{mathtools, nccmath}
\usepackage{wrapfig}
\usepackage{comment}

% To generate dummy text
\usepackage{blindtext}


%color
%\usepackage[dvipsnames]{xcolor}
% \usepackage{xcolor}


%\usepackage[pdftex]{graphicx}
\usepackage{graphicx}
% declare the path(s) for graphic files
%\graphicspath{{../Figures/}}

% extensions so you won't have to specify these with
% every instance of \includegraphics
% \DeclareGraphicsExtensions{.pdf,.jpeg,.png}

% for custom commands
\usepackage{xspace}

% table alignment
\usepackage{array}
\usepackage{ragged2e}
\newcolumntype{P}[1]{>{\RaggedRight\hspace{0pt}}p{#1}}
\newcolumntype{X}[1]{>{\RaggedRight\hspace*{0pt}}p{#1}}

% color box
\usepackage{tcolorbox}


% for tikz
\usepackage{tikz}
%\usetikzlibrary{trees}
\usetikzlibrary{arrows,shapes,positioning,shadows,trees,mindmap}
% \usepackage{forest}
\usepackage[edges]{forest}
\usetikzlibrary{arrows.meta}
\colorlet{linecol}{black!75}
\usepackage{xkcdcolors} % xkcd colors


% for colorful equation
\usepackage{tikz}
\usetikzlibrary{backgrounds}
\usetikzlibrary{arrows,shapes}
\usetikzlibrary{tikzmark}
\usetikzlibrary{calc}
% Commands for Highlighting text -- non tikz method
\newcommand{\highlight}[2]{\colorbox{#1!17}{$\displaystyle #2$}}
%\newcommand{\highlight}[2]{\colorbox{#1!17}{$#2$}}
\newcommand{\highlightdark}[2]{\colorbox{#1!47}{$\displaystyle #2$}}

% my custom colors for shading
\colorlet{mhpurple}{Plum!80}


% Commands for Highlighting text -- non tikz method
\renewcommand{\highlight}[2]{\colorbox{#1!17}{#2}}
\renewcommand{\highlightdark}[2]{\colorbox{#1!47}{#2}}

% Some math definitions
\newcommand{\lap}{\mathrm{Lap}}
\newcommand{\pr}{\mathrm{Pr}}

\newcommand{\Tset}{\mathcal{T}}
\newcommand{\Dset}{\mathcal{D}}
\newcommand{\Rbound}{\widetilde{\mathcal{R}}}




\newcommand{\code}[1]{\lstinline|#1|}





\begin{document}

\title{{基于Julia的单细胞}{数据分析平台的设计与实现}}

\entitle{{Design and Implementation of a Julia-based}{Single-Cell Data Analysis Platform}}

\author{李泽华}
\advisor{赵伟}
\major{生物信息学}
\college{生命科学学院}
\grade{2021级}



\maketitle



%==============================%
% ↓ ↓ ↓ 诚信说明页 授权说明书
%==============================%

% 1. 可以调整签字的宽度,现在是40
% 2. 去掉raisebox的相关注释(注意上下大括号对应),可以改变-5那个数字调整签名和横线的上下位置

% 你的签名,signature.pdf 改为你的签名文件名,
\mysignature{
    % \raisebox{-5pt}{
    \includegraphics[width=40pt]{img/my_signature.png}
    % }
}
% 你手写的日期,signature.pdf 改为你的手写的日期文件名
\mytime{2025年4月27日}
% \mytime{
%     % \raisebox{-5pt}{
%     \includegraphics[width=40pt]{img/time_signature.pdf}
%     % }
% }
% 老师的手写签名,signature.pdf 改为老师的手写签名文件名
\supervisorsignature{
    % \raisebox{-5pt}{
    \includegraphics[width=40pt]{img/teacher_signature.png}
    % }
}
% 老师手写的时间,signature.pdf 改为老师的手写的日期文件名
\teachertime{2025年4月30日}
%     % \raisebox{-5pSt}{
%     \includegraphics[width=40pt]{img/teacher_time_signature.pdf}
%     % }
% }
% 老师手写的成绩
% \recommendedgrade{
%     % \raisebox{-5pt}{
%     \includegraphics[width=40pt]{img/score.pdf}
%     % }
% }

\makestatement

%==============================%
% ↑ ↑ ↑ 诚信说明页 授权说明书
%==============================%


%=====%
%论文(设计)成绩:注意2007的模板要求,成绩页在最后,2021要求成绩页在摘要前面
%=====%

\supervisorcomment{导师评价}
\recommendedgrade{0}

% \committeecomment{优秀}

% \finalgrade{100}
% 上面这些注释掉可以去掉成绩、评语什么的



\frontmatter



%中文摘要
\ZhAbstract{
单细胞转录组测序(scRNA-seq)等单细胞技术正以前所未有的速度飞跃发展,推动生命科学进入以细胞为基本单元深入解析的新时代。相较于传统的群体测序(bulk RNA-seq),单细胞测序能够捕捉个体细胞之间的异质性,揭示更为精细的生物学图景。然而,由于数据的高维性、稀疏性及噪声等特性,其分析对计算工具与算法提出了更高要求。

目前,Scanpy(Python)、Seurat(R)等主流分析平台虽功能完备,但在性能方面受限于解释性语言的运行效率。为追求高性能,它们往往依赖 C/C++ 或 Rust 等静态类型语言进行底层加速,这不仅加重了使用负担,也提高了调试的门槛。

本研究提出 Juscan.jl 是一个完全使用 Julia 编写的单细胞数据分析平台,旨在打破性能瓶颈与开发复杂度之间的权衡。基于 Muon.jl 提供的 AnnData 数据结构,Juscan.jl 实现了包括数据预处理、高变基因识别、降维与聚类在内的一站式分析流程。平台设计力求模块化与简洁性,API 风格对标 Scanpy,降低用户迁移成本,并可充分利用 Julia 的 JIT 编译与原生并行机制,实现高性能计算。

实验结果表明,在大规模数据集处理上,Juscan.jl 的效率明显优于 R 语言的 Seurat,虽初版性能略低于 Scanpy,但 Julia 语言本身具备更高的性能上限,随着后续优化有望显著提升。本研究不仅为 Julia 在单细胞领域的应用奠定基础,也为高性能生物数据分析提供了新思路与工具支持。
}{
单细胞转录组测序(scRNA-seq), 生物信息学, Julia, 高性能计算, 数据分析工具
}


%英文摘要
\EnAbstract{
Single-cell technologies, such as single-cell RNA sequencing (scRNA-seq), are advancing at an unprecedented pace, ushering life sciences into a new era of analysis at cellular resolution. Compared to traditional bulk RNA sequencing, single-cell approaches enable the capture of cellular heterogeneity, revealing a much more refined landscape of biological processes. However, due to the high dimensionality, sparsity, and inherent noise of such data, their analysis places stringent demands on computational tools and algorithmic frameworks.

Currently, mainstream platforms like Scanpy (Python) and Seurat (R) offer comprehensive functionalities but are limited in performance due to the runtime inefficiencies of interpreted languages. To overcome these limitations, many of these tools rely on static languages such as C/C++ or Rust for backend acceleration—an approach that increases development complexity and raises the barrier for debugging and maintenance.

In this study, we introduce Juscan.jl, a single-cell data analysis platform written entirely in Julia, aiming to break the trade-off between computational performance and development simplicity. Built upon the AnnData structure provided by Muon.jl, Juscan.jl delivers an end-to-end pipeline for single-cell analysis, encompassing data preprocessing, highly variable gene selection, dimensionality reduction, and clustering. The platform adopts a modular and minimalist design, with an API style inspired by Scanpy to ease user migration, while fully leveraging Julia’s just-in-time (JIT) compilation and native parallelism for high-performance computing.

Experimental results show that Juscan.jl significantly outperforms Seurat when processing large-scale datasets. While its current version slightly trails Scanpy in performance, Julia's inherently higher performance ceiling offers promising potential for future optimization. This work not only lays a foundation for Julia’s application in single-cell transcriptomics but also offers new perspectives and tools for high-performance biological data analysis.
}{
Single-cell RNA sequencing (scRNA-seq), bioinformatics, Julia, high-performance computing, data analysis tools
}

%生成目录
\customcontent

%文章主体
\mainmatter
%=================================================================

\chapter{引言}
%%% 引言

\section{研究背景}
\subsection{单细胞数据分析及其在生物学的重要性}

著名细胞学家E. B. Wilson 曾在他的著作 The Cell in Development and Heredity中强调"The key to every biological problem must finally be sought in the cell.",即便是生物化学,分子生物学,遗传学高度发达的今天,我们不但在研究几乎任何生命现象的过程中都离不开细胞的基本单位,而且从细胞的角度进行研究比以往任何时间都有现实意义。

在 Wilson 提出该观点的年代,生物学的研究主要依靠显微镜和简单的生化实验来探讨细胞和亚细胞单位的异质性和功能,这一过程主要依赖形态学进行分析,难以揭示细胞的分子差异。而后出现的群体测序(bulk RNA-seq),也只能检测细胞群体的平均状态,无法区分单个细胞的差异性。

随着分子生物学的发展,人类逐渐了解了DNA,RNA的结构,并可以通过桑格测序,二代测序方法获取核酸序列信息。现代的单细胞分析技术,使用优化的二代测序技术,以较高的单细胞分辨率,检查单个细胞的核酸序列。以转录组测序(scRNA-seq)为例,首先使用荧光激活细胞分选术(FACS)或微流体设备\upcite{Poulin2016}高通量的快速分离数百至数万个细胞,然后通过与群体测序(bulk RNA-seq)类似的步骤进行测序,即逆转录(RT)、扩增、文库生成和测序\upcite{Li2021},这里不做展开。

可以看到,单细胞技术使得 Wilson 的观点进入了可操作的时代。单细胞转录组测序(scRNA-seq)、单细胞表观遗传测序(ATAC-seq)、空间转录组学(spatial transcriptomics)等方法,使得现代的研究可以真正在单个细胞的分辨率下探索生物学问题。

对于单细胞测序获得的大规模数据,需要通过生物信息学的分析方法研究。单细胞数据分析包括从数据预处理、降维、聚类、伪时序分析到多组学整合的多个环节,现有许多计算工具来支持这一研究领域。其中,Scanpy\upcite{ScanpySingleCellAnalysis}、Seurat、AnnData 以及 SingleCellExperiment 等成为主流,推动了单细胞数据的标准化并降低了数据分析门槛。
% 此外,随着计算方法的进步,基于深度学习和多组学整合的工具也不断发展,为单细胞数据分析提供了更多可能性。

\subsection{单细胞数据分析概述}

% 大多数基因仅用于一小部分细胞类型,但是由于 scRNA-seq 实验中通常使用的起始材料量少且测序深度低,某些基因即使表达也会无法检测到。结果是基因表达矩阵中出现大量零值,这是有问题的,因为其中一些零可能表示细胞中实际的低表达或零表达以及测量过程的变化10。区分和适当建模这些观察到的零源的困难是计算分析面临的主要挑战之一。即使是深度测序的数据集也可能有约 50\% 的零,而浅度测序的数据集可能有 99\% 的零。相比之下,在典型的批量 RNA 测序数据集中,不到 20\% 的数据条目是零。

scRNA-seq是一套流行而强大的单细胞检测和分析技术,可以让研究者以单个细胞为单位获取并分析大量细胞整个转录组。但这同时也带来了巨大的数据集,需要使用专业的统计和分析流程。scRNA-seq分析围绕表达矩阵展开,即\lstinline|Anndata.X|,代表了每个基因和细胞中观察到的转录本数量。以表达矩阵为中心进行数学建模设计和搭建一整套统计和分析算法。

\subsubsection{(1)数据预处理}

在开始真正的数据分析之前,为了保证下游分析结果的准确性,需要对数据进行基础的预处理。庞大的表达矩阵中,由于技术精度等问题,优质的数据与噪音,杂质等夹杂在一起。可以通过设定一个数据集特定的 UMI 数量阈值来过滤。确保只保留那些显著偏离背景水平的条形码,从而捕获那些 RNA 含量较低但真实存在的细胞。

但这种初步的筛选无法排除受损和死亡的细胞,此时就需要通过计算一些质量控制指标来识别,这些指标通常包括每个细胞中表达的基因数,每个基因在多少个细胞中被表达,线粒体来源RNA的比例以及无法比对或多重比对的reads比例等。

此外,对于scRNA-seq来说,无论是高通量低深度的10X方法还是低通量高深度的Smart-seq2协议,因为每个细胞的 RNA 量会因细胞周期阶段和其他生物因素而有很大差异,导致从测序实验中获得的有用读数的数量差异显著。为了矫正这种差异,会对整体的scRNA-seq数据计算一个与样本测序升读相关的量,即$size facter$, 然后令整个表达矩阵除以该值。

\subsubsection{(2)高变异基因}

识别高变异基因 (highly variable genes/HVG)是为了通过过滤掉信息量较少的特征来解决高维数据中固有的维数灾难,方便包括降维,聚类,伪时序分析等在内的下游分析,使算法能够专注于最具信息量的特征来描述细胞的异质结构。也因此成为了决定分析效果的重要步骤。

% 原始计数值被用来构建一个对数方差与对数均值之间的拟合曲线,基于这条曲线,系统能精准地识别出那些偏离常规变异模式的基因。

Cell Ranger 方法通过对数归一化后的表达值计算离散度,高离散度的基因被选中作为HVGs,展现出它们在细胞间独特的变异性。而根据seurat的经验,seurat\_v1和seurat\_v2使用归一化表达矩阵在计算离散度,通过不同的离散度计算方法,筛选出表达矩阵中高变异的基因。而seurat\_v3使用原始计数构建出一个对数方差和对数均值之间的拟合曲线,通过曲线精准的识别偏离常规编译模式的基因。

\subsubsection{(3)降维}

减少表达矩阵高维性负面影响的另一种策略是在减少的特征空间上执行降维。其中最常用的方法是PCA,这是一中线性降维方法,能够保留细胞间的欧几里得距离,并在大规模数据集中迅速计算出重要的成分,也就是方差最大的方向。需要保留的维度数是PCA的一个重要参数,取决于数据的复杂度,常见的方法是根据“肘部法则”,绘制每个成分解释的方差比例图,选择曲线上肘部的点筛选出最具代表性的成分。

然而即便如此,单独的几个主成分仍然难以代表scRNA-seq数据的全部表达矩阵,特别是对于复杂、弯曲的流形结(manifold structure),因此,需要依靠可视化算法将多维信息映射到二维平面上。当前最流行的就是Uniform Manifold Approximation and Projection(UMAP),它构建细胞间的近邻网络,来近似模拟数据的拓扑结构,再使用高斯核函数计算邻接概率,如公式\ref{eq:pneighbor}所示,并通过最小化一个交叉熵(Cross Entropy)形式的损失函数即公式\ref{eq:loss},在低维空间中模拟重现高维结构。

\begin{equation}
  \label{eq:pneighbor}
  p_{ij} = \exp\left(-\frac{||x_i - x_j||^2}{\sigma_i}\right)
\end{equation}

\begin{equation}
  \label{eq:loss}
  L = \sum_{i \ne j} \left[ p_{ij} \log \left( \frac{p_{ij}}{q_{ij}} \right) + (1 - p_{ij}) \log \left( \frac{1 - p_{ij}}{1 - q_{ij}} \right) \right]
\end{equation}

其中$p_{ij}$表示低维空间中点$i$和点$j$由sigmoid-like函数计算得来的相似度。

另一个常见的方法是t-distributed stochastic neighbor embedding(t-SNE),但他在保留大规模数据上的能力不足UMAP。同时需要注意的是,不论是UMAP还是t-SNE都需要用户提供的的超参数,并且运算结果对该参数非常敏感。

\subsubsection{(4)聚类}

细胞聚类是一个把细胞集合划分成子集的过程,每个子集是一个簇,使得簇中的对象彼此相似,但与其他簇中对象彼此相异。

数据聚类等无监督学习随着单细胞测序技术的进步,已经成为识别细胞类型和发现新细胞的主要方法。当前生物信息学针对单细胞RNA序列数据已经开发了多种类型的聚类方法,主要分为k均值聚类,层次聚类,社区发现以及基于密度的聚类\upcite{zhang2023review}。下面的两个表\ref{tab:clusterSummary1}, \ref{tab:clusterSummary2}总结了当前常用的聚类方法的优缺点以及特点。


% \begin{longtable}{>{\centering\arraybackslash}p{2cm}
%                   >{\centering\arraybackslash}p{2.5cm}
%                   >{\centering\arraybackslash}p{5cm}
%                   >{\centering\arraybackslash}p{5cm}}
%   \caption{单细胞 RNA-seq 数据中先进聚类方法概述1}
%   \label{tab:clusterSummary1} \\
%   \toprule
%   \textbf{方法} & \textbf{类型} & \textbf{优点} & \textbf{缺点} \\
%   \midrule
%   \endfirsthead
%   
%   \toprule
%   \textbf{方法} & \textbf{类型} & \textbf{优点} & \textbf{缺点} \\
%   \midrule
%   \endhead
%   
%   \bottomrule
%   \endfoot
%   
%   \scriptsize SAIC      & \scriptsize k-means            & \scriptsize 复杂度低;可扩展至大数据规模       & \scriptsize 对离群值敏感;无法估计聚类数量 \\
%   \scriptsize RaceID    & \scriptsize k-means            & \scriptsize 对稀有细胞类型敏感;可估计聚类数量 & \scriptsize 不适用于稀有细胞类型 \\
%   \scriptsize pcaReduce & \scriptsize k-means/层次聚类   & \scriptsize 可提供层次结构的聚类结果           & \scriptsize 不稳定;不适用于大规模数据 \\
%   \scriptsize SC3       & \scriptsize k-means/层次聚类   & \scriptsize 准确性高;可估计聚类数量           & \scriptsize 复杂度高 \\
%   \scriptsize CIDR      & \scriptsize 层次聚类           & \scriptsize 对 dropout(数据缺失)敏感         & \scriptsize 复杂度高 \\
%   \scriptsize BackSPIN  & \scriptsize 层次聚类           & \scriptsize 可同时对基因和细胞进行聚类         & \scriptsize 复杂度高 \\
%   \scriptsize SIMLR     & \scriptsize 谱聚类(Spectral) & \scriptsize 适用于异质性和噪声数据             & \scriptsize 不适用于大规模数据 \\
%   \scriptsize SinNLRR   & \scriptsize 谱聚类(Spectral) & \scriptsize 适用于噪声数据                     & \scriptsize 无法估计聚类数量 \\
%   \scriptsize SCANPY    & \scriptsize Louvain            & \scriptsize 复杂度低;可扩展至大数据规模       & \scriptsize 可能无法识别小的群体 \\
%   \scriptsize Seurat    & \scriptsize Louvain            & \scriptsize 复杂度低;可扩展至大数据规模       & \scriptsize 可能无法识别小的群体 \\
%   \scriptsize GiniClust & \scriptsize 基于密度的方法     & \scriptsize 可检测稀有细胞类型                 & \scriptsize 对大型聚类不敏感 \\
% \end{longtable}

\begin{longtable}{>{\centering\arraybackslash}p{2cm}
                  >{\centering\arraybackslash}p{2.5cm}
                  >{\centering\arraybackslash}p{5cm}
                  >{\centering\arraybackslash}p{5cm}}
  \caption{单细胞 RNA-seq 数据中先进聚类方法概述1}
  \label{tab:clusterSummary1} \\
  \toprule
  \textbf{方法} & \textbf{类型} & \textbf{优点} & \textbf{缺点} \\
  \midrule
  \endfirsthead
  
  \toprule
  \textbf{方法} & \textbf{类型} & \textbf{优点} & \textbf{缺点} \\
  \midrule
  \endhead
  
  \bottomrule
  \endfoot
  
  SAIC      & k-means            & 复杂度低;可扩展至大数据规模       & 对离群值敏感;无法估计聚类数量 \\
  RaceID    & k-means            & 对稀有细胞类型敏感;可估计聚类数量 & 不适用于稀有细胞类型 \\
  pcaReduce & k-means/层次聚类   & 可提供层次结构的聚类结果           & 不稳定;不适用于大规模数据 \\
  SC3       & k-means/层次聚类   & 准确性高;可估计聚类数量           & 复杂度高 \\
  CIDR      & 层次聚类           & 对 dropout(数据缺失)敏感         & 复杂度高 \\
  BackSPIN  & 层次聚类           & 可同时对基因和细胞进行聚类         & 复杂度高 \\
  SIMLR     & 谱聚类(Spectral) & 适用于异质性和噪声数据             & 不适用于大规模数据 \\
  SinNLRR   & 谱聚类(Spectral) & 适用于噪声数据                     & 无法估计聚类数量 \\
  SCANPY    & Louvain            & 复杂度低;可扩展至大数据规模       & 可能无法识别小的群体 \\
  Seurat    & Louvain            & 复杂度低;可扩展至大数据规模       & 可能无法识别小的群体 \\
  GiniClust & 基于密度的方法     & 可检测稀有细胞类型                 & 对大型聚类不敏感 \\
\end{longtable}

\begin{ThreePartTable}
\begin{longtable}{ccccc}
\caption{不同聚类方法的特性比较} \label{tab:clustering_methods} \\

\toprule
\textbf{方法} & \textbf{复杂度} & \textbf{可自动确定聚类数} & \textbf{可扩展性} & \textbf{可检测稀有聚类} \\
\midrule
\endfirsthead

\toprule
\textbf{方法} & \textbf{复杂度} & \textbf{可自动确定聚类数} & \textbf{可扩展性} & \textbf{可检测稀有聚类} \\
\midrule
\endhead

\midrule
\insertTableNotes
\endfoot

\bottomrule
\insertTableNotes
\endlastfoot

SAIC      & $O(N \cdot K \cdot T)$          & \XSolidBrush & \checkmark   & \XSolidBrush \\
RaceID    & $O(N \cdot K \cdot T)$          & \checkmark   & \checkmark   & \checkmark   \\
pcaReduce & $O(N \cdot K \cdot T) / O(N^3)$ & \checkmark   & \XSolidBrush & \XSolidBrush \\
SC3       & $O(N \cdot K \cdot T) / O(N^3)$ & \checkmark   & \XSolidBrush & \XSolidBrush \\
CIDR      & $O(N^3)$                        & \checkmark   & \checkmark   & \checkmark   \\
BackSPIN  & $O(N^3)$                        & \XSolidBrush & \XSolidBrush & \checkmark   \\
SIMLR     & $O(N^3)$                        & \checkmark   & \XSolidBrush & \XSolidBrush \\
SinNLRR   & $O(N^3)$                        & \checkmark   & \XSolidBrush & \XSolidBrush \\
SCANPY    & $O(N \log N)$                   & \checkmark   & \checkmark   & \XSolidBrush \\
Seurat    & $O(N \log N)$                   & \checkmark   & \checkmark   & \XSolidBrush \\
GiniClust & $O(N \log N)$                   & \XSolidBrush & \checkmark   & \checkmark   \\
\end{longtable}
\begin{TableNotes}
\small
\item 注:$N$ 表示样本数量,$K$ 表示聚类簇数,$T$ 表示迭代次数。时间复杂度 $O$, 描述算法运行所需的时间,通常取决于输入数据的规模 $n$。

\end{TableNotes}

\end{ThreePartTable}

a.~k均值聚类

k-means是流行的聚类方法中最简单的,需要预定义聚类中心,首先随机初始化每个单元的质心。通过不断迭代来最小化质心与其单元各点之间的欧氏距离的平方和。
此方法可以随着数据点的数量线性拓展,因此适用于大型数据。但问题在于k-means是一种贪婪算法,容易陷入局部最优解,而且选择较大的k值, 会提高分群的准确度, 但同时会增大过拟合的风险。

b.~社区发现

社区发现算法在社会学和生物学数据这样可以被表示为具有节点和边的图系统中是一种十分流行的聚类算法。
对于单细胞RNA数据,节点指的是细胞,边权重为细胞之间在表达空间里的亲近度。Louvain是最流行的社区发现算法,广泛用于单细胞RNA序列数据\upcite{blondel2008fast},常用的Scanpy,Seurat包都提供了Louvain的聚类方法对单细胞进行聚类。
专注于优化模块度(modularity)函数,揭示图中高度连接的子结构——即“社区”。

Louvain递归地将社区合并为单个节点,即先对原始图进行社区划分,找到局部模块度最优的社区结构,接着,把每一个识别出来的社区看作是图中的一个特殊节点,将整张图重新构造为一个新的压缩图。接下来在压缩图上执行模块度聚类,再次执行模块度优化算法,寻找新的社区划分。Louvain的时间复杂度为$O(N \log N)$,显著低于其他社区发现算法。

% \begin{equation}
% H=\frac{1}{2m}\sum_c\left[ e_c-\gamma\frac{k_c^2}{2m} \right]
% \end{equation}
% 其中m为总边数, $e_c$为社区C内的边数, $k_c$为社区C中节点数的和, $\gamma$为参数分辨率

\begin{figure}[h]
    \vspace*{1.8\baselineskip}
    \begin{equation}
        \label{eq:modularity}
        \hspace*{-6em}
        H=\frac{1}{2\tikzmarknode{js}{\highlight{red}{m}}}\sum_{\tikzmarknode{lmax}{\highlight{OliveGreen}c}}\left[ \tikzmarknode{ec}{\highlight{purple}{$e_c$}}-\tikzmarknode{ga}{\highlight{NavyBlue}{$\gamma$}}\frac{\tikzmarknode{kc}{\highlight{bittersweet}{$k_c$}}^2}{2\tikzmarknode{jd}{\highlight{red}{m}}} \right]
        % H=\frac{1}{2m}\sum_c\left[ e_c-\gamma\frac{k_c^2}{2m} \right]
    \end{equation}
    \vspace*{0.4\baselineskip}
    \begin{tikzpicture}[overlay,remember picture,>=stealth,nodes={align=left,inner ysep=1pt},<-]
        % m
        \node[anchor=north,color=red!57, yshift=-1.2em] (jtext) at ($(js.south)!0.5!(jd.south)$) {\textsf{\footnotesize total number of edges}};
        \draw[<->,color=red!57] (js.south) |- (jtext.north) -| (jd.south);
        % c
        \path (lmax.north) ++ (-2.7,-2.3em) node[anchor=north,color=xkcdHunterGreen!85] (lmaxtext){\textsf{\footnotesize total number of nodes}};
        \draw [color=xkcdHunterGreen](lmax.south) |- ([xshift=-0.4ex,color=xkcdHunterGreen]lmaxtext.south west);
        % ec
        \path (ec.north) ++ (0,1.8em) node[anchor=south east,color=Plum!85] (ntext){\textsf{\footnotesize intra-community edge count of $C$}};
        \draw [color=Plum](ec.north) |- ([xshift=-0.3ex,color=Plum]ntext.south west);
        % gamma
        \path (ga.north) ++ (0,4em) node[anchor=north west,color=NavyBlue!85] (mitext){\textsf{\footnotesize resolution parameter}};
        \draw [color=NavyBlue](ga.north) |- ([xshift=-0.3ex,color=NavyBlue]mitext.south east);
        % kc
        \path (kc.north) ++ (0,1.9em) node[anchor=north west,color=Bittersweet!85] (lijtext){\textsf{\footnotesize total degree of nodes in $C$}};
        \draw [color=Bittersweet](kc.north) |- ([xshift=-0.3ex,color=Bittersweet]lijtext.south east);
    \end{tikzpicture}
  % \end{minipage}
\end{figure}


如公式\ref{eq:modularity}所示,模块度$H$是一个度量图结构中社区划分优劣的指标,越高表示社区划分越合理。

\subsection{现有单细胞分析平台}

近年来,随着单细胞 RNA 测序(scRNA-seq)技术的快速发展,一系列面向单细胞数据分析的计算工具应运而生,并在科研社区中建立起广泛的用户基础。其中,Scanpy(Python)、Seurat(R) 以及 SingleCellExperiment(R) 是目前应用最为广泛的三个分析平台,各自具备鲜明的设计理念与技术优势。

\textbf{(1)~Seurat}

Seurat\supercite{hao_integrated_2021,hao_dictionary_2024} 是 R 语言中最具代表性的单细胞分析工具之一,自发布以来持续更新,目前已发展至第五版。它采用模块化设计,涵盖从质量控制、归一化、特征选择,到聚类、降维与可视化的完整流程。Seurat 的一大特色是其对复杂生物结构的建模能力,尤其是在样本整合、空间转录组、轨迹推断等方向中具有领先优势。此外,Seurat 提出了如“anchor”、“dictionary learning”等创新性算法,为多组学与参考映射提供了解决方案。

然而,Seurat 的运行效率相对受限,尤其在处理大型数据集时存在内存占用高、执行缓慢的问题。尽管开发团队通过引入 C++ 加速模块进行优化,但其主框架依然依赖于解释性语言 R,在高性能计算方面存在一定局限。

\textbf{(2)~scanpy}

Scanpy\upcite{wolf_scanpy_2018} 是 Python 语言中最主流的单细胞分析工具,构建在 AnnData 数据结构之上,具有良好的可扩展性与模块兼容性。其设计理念强调“轻量级、高性能、面向大数据”,适合处理十万级以上细胞的 scRNA-seq 数据。同时,Scanpy 与 Muon 框架联合提供了对多模态数据(如 RNA+ATAC)的支持\upcite{bredikhinMUONMultimodalOmics2022}。

相比 Seurat,Scanpy 拥有更快的处理速度和更自由的编程能力,但其函数接口相对低层,用户需具备较强的编程背景。更重要的是,Scanpy 在底层实现中大量调用了 Numba、Cython 和 C++ 等异构技术,尽管提升了性能,但也增加了调试难度和系统依赖复杂性。

\textbf{(3)~SingleCellExperiment}

SingleCellExperiment\upcite{amezquita_orchestrating_2020} 并非一个完整分析工具,而是 R/Bioconductor 生态中用于组织单细胞数据的标准数据结构。它以 SummarizedExperiment 为基础,提供统一的容器形式用于存储表达矩阵、元数据和降维结果。该结构被广泛用于支持 Bioconductor 生态中的各类单细胞工具,如 scater、scran、batchelor 等,是构建可复现分析流程的重要基础。

与 Scanpy 和 Seurat 相比,SingleCellExperiment 更注重数据结构的标准化和模块解耦,适合构建灵活、可插拔的分析工作流。然而,它缺乏统一的“主框架”,用户需组合多个包使用。

\section{研究意义与目标}
\subsection{研究意义}

近年来, 由于python语言的简单易用, 以及其丰富的第三方库, 使得python成为了数据分析领域最热门的语言。因而基于python产生了众多单细胞库, 例如anndata\upcite{bredikhinMUONMultimodalOmics2022}, scanpy\upcite{ScanpySingleCellAnalysis}。

但同时, python在也因为其极低的运行效率饱受诟病。这主要是因为其动态类型, 解释运行和难以多核并发等特性所导致的。
从传统意义上来说, 较高的运行效率是以牺牲开发效率为代价的, 例如C/C++,
但是随着编程语言技术的发展, 一种新的编译器框架技术即LLVM(Low-Level Virtual Machine)的出现, 使得计算机可以在不牺牲开发效率的情况下获得较高的运行效率。

Julia\supercite{SponsorJuliaLangGitHub,roeschJuliaBiologists2023}是一门新兴的数据科学语言, 使用LLVM作为其核心编译器基础设施, 它为julia提供了一个高度优化的编译器框架, 使Julia能够将代码编译成高效的机器码, 从而提供接近C和Fortran的运行性能。
Julia还引入了JIT(Just-In-Time)即时编译技术, 由于LLVM支持即时(JIT)编译, Julia可以在运行时进行动态编译。
这意味着, Julia是一种动态的强类型语言, 变量一旦被赋予某种类型, 该类型在运算过程中不会改变, 编译器可以在编译期间推断出变量和表达式的具体类型, 同时JIT编译器可以在生成代码时进行高度优化, 避免诸如类型检查或类型转换等操作。
另外, python的GIL(Global Interpreter Lock)机制使其无法充分利用多核CPU, 而Julia可以充分并方便利用多核CPU, 从而提高运行效率, 特别是对于单细胞处理中质控, 降维, 聚类这些CPU密集型任务。

但在 Julia 生态系统中,单细胞分析工具尚处于初步发展阶段,社区尚未形成完善的支持体系,第三方库的数量和功能也相对有限。

目前,Julia 平台上已有一些小型的单细胞分析库。例如,Muon.jl 是由 Anndata 官方团队开发的 Julia 版本,旨在提供与 Python 中的 Muon 类似的数据结构。然而,Muon.jl 主要专注于数据结构的实现,缺乏完整的分析工具链,用户需要自行实现数据预处理、降维、聚类等分析步骤。

此外,Julia 社区中还存在一些功能较为单一的库,专注于单细胞分析的特定方面。例如,scVI.jl\upcite{scVIjl} 是一个用于拟合变分自编码器(VAE)模型的 Julia 包,基于 Python 的 scvi-tools 实现,主要用于处理计数分布的单细胞数据。该库提供了标准和线性解码的 VAE 模型\upcite{lopez_deep_2018},支持负二项分布、泊松分布、高斯分布和伯努利分布等生成分布,以及不同的离散参数设置。

还有一些库,如 ASCT.jl\upcite{Yang2023.12.27.573479},虽然实现了类似 Seurat 的功能,提供了从质量控制、预处理、降维、聚类、标记基因识别到样本整合的完整流程,但并未基于官方的 Muon.jl 结构,而是采用了自定义的数据结构。这种做法可能导致与其他工具的兼容性问题,限制了数据的共享和复用。

综上所述,虽然 Julia 在性能和语法方面具备显著优势,但其在单细胞分析领域的工具生态仍不够成熟,缺乏功能全面、结构统一的分析平台。正因为如此,开发一个基于 Julia 的全栈单细胞分析工具,既是对现有生态的补足,也是推动高性能生物信息学工具发展的关键契机。

随着测序技术的飞速进步,单细胞数据的规模持续膨胀,传统工具在分析效率上的瓶颈逐渐显现。高性能工具的缺位带来了计算资源浪费与能源开销的增加,成为科研中的隐性成本。而 Julia 所带来的运行效率提升,不仅可以缩短实验周期,加快结果反馈,有助于快速迭代实验与建模,也意味着更低的资源消耗,节省服务器运行成本,降低科研预算压力。

因此,构建一个拥有 Scanpy/Seurat 等平台的数据结构规范,同时具备 Julia 本地性能优势的单细胞分析库,不仅能够提升分析效率,更将有助于推动大规模、深层次的生命科学研究向前迈进,为多模态组学、空间转录组、个体化医疗等新兴方向奠定计算基础。

% \section{研究目标与内容}
\subsection{本项目的目标}

本项目的主要目标是用julia语言实现一个由scanpy启发的数据分析库,用于scRNA-seq的分析,弥补julia生态在单细胞分析上的不足。

在项目稳步推进并完成一个可用版本的前提下 ,利用Julia在数值计算和并行计算方面的优势,保证api的高效性和性能,以实现对大规模数据集的分析工作。API的设计应符合julia官方要求,并尽量与python保持一致,本提供完整的手册与官方文档,以降低用户从scanpy迁移到Juscan.jl的学习成本。另外构建过程采用模块化设计并与Julia生态中的其他工具无缝对接,在提高代码的可读性的同时方便未来的迭代并集成更多高级功能。



\chapter{系统设计与实现}
\section{衰老与秀丽隐杆线虫}

衰老是生命活动中一种极为复杂的生物学现象,它贯穿生命的整个历程。从分子层面 的微观变化,到器官乃至整个系统的宏观转变,无一不彰显着这一过程的深远影响。在生 物体衰老的过程中,细胞功能会逐渐衰退,细胞的代谢活性也随之降低。与此同时, 氧化应 激会不断在体内积累\upcite{lopez-otin_hallmarks_2023}。此外, DNA 修复能力也会逐渐丧失, 这些变化相互交织、相互影 响,最终使得生物个体的生存能力逐渐降低、适应能力下降、生命走向衰退\upcite{noauthor__nodate-1}。近年来, 衰 老研究逐渐成为生物医学领域的热点,不仅因为衰老是许多慢性疾病的主要风险因素,而 且它是人类健康寿命延长的重要方向\upcite{olshansky_implausibility_2024}。为了深入探索衰老的分子机制,科学家们开发了 多种模型生物,其中秀丽隐杆线虫因其独特的优势成为研究衰老的重要工具\upcite{jeayeng_caenorhabditis_2024} (图\ref{fig:worm_aging})。

\begin{figure}[H]
    \centering
    \includegraphics[width=0.8\textwidth]{img/worm_aging.jpg}
    \caption{线虫结构以及线虫衰老过程中的生理变化:(a) 线虫基本的身体数据;(b) 线虫在衰老过程中的生理变化}
    \label{fig:worm_aging}
\end{figure}

秀丽隐杆线虫因其生命周期短、遗传背景简单、易于操作以及完全测序的基因组等特 点,已成为研究衰老和发育生物学的理想模型生物\upcite{jeayeng_caenorhabditis_2024}。在秀丽隐杆线虫的衰老研究中,成 虫期从其被孵化后的第 1 天开始算起,其生殖期通常处于第 3-4 天,而衰老的迹象则从线 虫的第 8 天左右开始显现\upcite{jeayeng_caenorhabditis_2024}(图\ref{fig:worm_aging})。因此在研究中, 第一天被视为线虫成虫期的起点, 第 八天被视为线虫发生明显的衰老的时间线。这种时间上的对应关系使得科学家能够通过对 线虫进行不同时间点的比较,研究衰老的动态过程。

\section{胞间连接及组织屏障}

胞间连接是细胞间进行通信和物质交换的重要途径。细胞间的三种连接方式为紧密连 接,粘附连接和间隙连接。其中,紧密连接发挥着渗透屏障作用,粘附连接负责固定、维 持组织中细胞间的相对位置,间隙连接则负责信号传递(图\ref{fig:cell_connection})。

\begin{figure}[H]
    \centering
    \includegraphics[width=0.8\textwidth]{img/three_cell_link_functions.jpg}
    \caption{细胞间连接的三种方式:紧密连接、粘附连接和间隙连接}
    \label{fig:cell_connection}
\end{figure}

在线虫中,胞间连接主要通过缝隙连接实现。缝隙连接是一种特殊的细胞间结构,允 许小分子在相邻细胞之间直接传递。这种连接在多种生理过程中起着关键作用,包括发育 调控、代谢协调和应激响应等\upcite{wang_structural_2024}。在线虫中, HMR-1 蛋白是一种重要的缝隙连接蛋白, 且 HMR-1 与哺乳动物的连接蛋白具有高度同源性。HMR-1 在线虫的多种组织中表达,包括 表皮、肠道和生殖器官等(图\ref{fig:HMR-1_protein})。研究表明, HMR-1 不仅参与了细胞间的直接通信, 还 在胚胎发育、组织屏障功能和环境应激反应中发挥重要作用\upcite{naturale_persistent_2023}。

\begin{figure}[H]
    \centering
    \includegraphics[width=0.8\textwidth]{img/HMR_worm.png}
    \caption{实验用绿色荧光蛋白标记 HMR-1 来显示线虫的表皮屏障示意图}
    \label{fig:HMR-1_protein}
\end{figure}

线虫的组织屏障主要包括表皮屏障和肠道屏障,这两种屏障在维持线虫的内环境稳定  和防御外界病原体侵害中起着关键作用(图\ref{fig:worm_barrier})。线虫的表皮屏障由外层的表皮细胞构成,

主要负责保护线虫免受外界机械损伤和病原体的侵害。表皮细胞通过紧密连接形成了一种 屏障结构,防止有害物质从外界进入线虫体内\upcite{zhang_structural_2015}。肠道屏障是线虫消化系统的重要组成部 分,主要由肠道上皮细胞构成。肠道屏障的主要功能是吸收食物中的营养物质, 同时防止 有害物质扩散进入线虫体内。线虫肠道上皮细胞通过紧密连接,发挥选择性通透性屏障的 作用,确保营养物质的吸收和有害物质的排除\upcite{tan_killing_1999}。

\begin{figure}[H]
    \centering
    \includegraphics[width=0.8\textwidth]{img/worm_blocks.png}
    \caption{线虫的组织屏障示意图}
    \label{fig:worm_barrier}
\end{figure}

胞间连接和组织屏障在功能上具有密切的联系。胞间连接的异常状态会导致组织屏障 的功能发生障碍, 进而影响线虫的整体生理状态\upcite{wang_structural_2024}。因此, 通过对线虫 HMR-1 蛋白和组织 屏障的研究,可以为开发新的临床治疗策略提供理论依据。

\section{组织蛋白酶}

\begin{figure}[H]
    \centering
    \includegraphics[width=0.5\textwidth]{img/protease.jpg}
    \caption{组织蛋白酶破坏细胞间连接示意图}
    \label{fig:protease}
\end{figure}

组织蛋白酶是一类能够特异性分解细胞外基质成分的蛋白酶,广泛参与组织修复、细胞迁移、胚胎发育和疾病发生\upcite{jakos_cysteine_2019} 等过程\upcite{wang_cathepsins_2023}。在线虫中, 组织蛋白酶在细胞屏障的维持、胞 间连接的调控以及生理和病理过程中都具有重要的功能\upcite{teuscher_longevity_2024}。组织蛋白酶可能会通过调节细 胞外基质的成分,间接影响胞间连接的结构和功能。然而,组织蛋白酶的过度表达可能导 致细胞连接的稳定性失调,破坏细胞间的通信以及组织屏障(图\ref{fig:protease})。

\section{研究不同种类之间同源物的意义}

在生物学种系发生理论中,若两个或多个结构具有相同的祖先,则称它们同源。这里 相同的祖先既可以指演化意义上的祖先, 即两个结构由一个共同的祖先演化而来, 比如,蝙 蝠的翅膀与人类的手臂是同源的;也可以指发育意义上的祖先, 即两个结构由胚胎时期的 同一组织发育而来,例如,人类女性的卵巢与男性的睾丸同源。总而言之,若两个或多个 基因、蛋白、结构、组织或器官等具有相同的祖先,则称它们互为同源物。

在生物学研究中, 同源物的研究具有多方面的重要意义。比如,科学家可以通过比较 同源基因或者蛋白质,确定物种之间的亲缘关系,构建进化树,预测基因功能以及进行基 因注释等。而通过研究不同物种之间的同源物,尤其是人类在不同模式生物中的同源物对 于开发针对人类不同疾病的治疗方法等极为关键。这种方法不仅高效有用,而且能够最大 程度上降低生物实验的成本,对于科研人员具有重要的科研价值和实际应用意义。

\section{RNAi实验在线虫中的应用}

RNAi干扰是一种分子生物学上由双链RNA诱发的基因沉默现象,其机制是通过阻碍  特定基因的转录或翻译来抑制基因表达。当细胞中导入与内源性mRNA编码区同源的双链 RNA时,该mRNA会发生降解从而导致基因表达沉默\upcite{mello_revealing_2004}。
	
在线虫中,RNAi 实验应用广泛\upcite{tabara_rde-1_1999}。它不但能够研究基因的功能,还能够确定信号通路 中不同基因的上下游关系,帮助我们理解细胞里的信息传递网络。线虫对 RNAi 反应敏感 高效,只需要通过喂食,就能轻松把 dsRNA送入线虫体内,引发 RNAi 反应。RNAi 实验在 线虫中的意义重大。它不仅推动了研究人员对线虫基因功能的认识,还为理解人类基因功 能和疾病机制提供了线索。而在我们的实验中之所以选择 RNAi 来敲低目的基因,是因为 我们的实验需要筛选目的基因,该试验便于进行大规模的基因筛选。

\section{小结}

本章节从衰老这一生命活动的复杂生物学现象出发, 阐述了秀丽隐杆线虫作为研究衰 老的重要模式生物的优势;进而深入探讨了胞间连接里紧密连接、粘附连接和间隙连接的 功能,且着重介绍了线虫缝隙连接及 HMR-1 蛋白的关键作用;接着又介绍了线虫组织屏障的构成与功能,以及胞间连接与组织屏障的密切联系;随后聚焦于组织蛋白酶, 阐述其 广泛参与多种生理病理过程的现象及其对胞间连接和组织屏障的潜在影响;最后介绍RNAi 实验原理及在线虫中的应用优势, 为后续基于 RNAi 的基因功能研究实验介绍奠定了 基础。


\chapter{实验与评估}
%%% 实验与评估

\section{实验数据集介绍}

本研究使用的单细胞转录组数据集为 10x Genomics 公司公开发布的经典示例数据集——PBMC 3k 数据集。该数据集包含约 2,700 个外周血单核细胞(Peripheral Blood Mononuclear Cells, PBMC),来自一位健康成年供体。PBMC 包括 T 细胞、B 细胞、自然杀伤细胞和单核细胞等多种免疫相关细胞类型,常作为免疫学研究中的模型系统。

PBMC 3k 数据集采用 10x Genomics Chromium 单细胞 3' RNA 测序平台进行文库构建,并使用 Illumina NextSeq 500 高通量测序仪进行测序。每个细胞平均测序深度约为 69,000 个 reads。原始数据使用 10x Genomics 官方分析工具 Cell Ranger(版本 1.1.0)进行预处理,包括细胞条形码识别、UMI 去重及基因表达矩阵生成。数据结果以稀疏矩阵形式存储,包含每个细胞在各个基因上的表达计数。

本数据集可以在\href{https://cf.10xgenomics.com/samples/cell/pbmc3k/pbmc3k_filtered_gene_bc_matrices.tar.gz}{10XGenomics官网}获取,并可以使用SeuratDisk,scDior等工具转换为h5ad文件。

由于数据来源权威、质量可靠、样本复杂性适中,PBMC 3k 数据集已广泛应用于 Seurat、Scanpy 等单细胞分析工具的功能验证与流程教学中,是验证聚类、降维、批次效应处理等算法性能的标准数据集之一。


\section{实验设计与流程}

本文基于 PBMC 3k 数据集,参考官方流程,实现了单细胞 RNA 测序数据分析的完整流程。以下是使用 Juscan.jl 库进行数据分析的示例:

\subsection{数据读取与质量控制}

Juscan.jl 库使用 Muon.jl 库作为数据存储的基础,支持读取 h5ad 格式的数据文件。数据读取后,使用\code{Juscan.Pp.filter_cells!} 和 \code{Juscan.Pp.filter_genes!} 函数进行细胞和基因的过滤,去除低质量细胞和低表达基因。接着,计算线粒体基因、核糖体基因和血红蛋白基因的表达比例,并根据这些指标进行进一步的质量控制。

\begin{fancybox}{Juscan.jl demo data load and quality control}
\addcontentsline{lot}{table}{代码~\thetcbcounter: Juscan.jl示例:数据读取与质量控制}
\begin{lstlisting}
using Juscan, Muon
using DataFrames, LinearAlgebra, SparseArrays

adata = Juscan.readh5ad("./data/pbmc_3k.h5ad")

############## quality control #################

Juscan.Pp.filter_cells!(adata, min_genes=200)
Juscan.Pp.filter_genes!(adata, min_cells=3)
Juscan.Pp.filter_cells!(adata, max_genes=2300)
Juscan.Pp.filter_cells!(adata, max_counts=10000)

adata.var.mt = startswith.(adata.var_names, "MT-")
adata.var.ribo = startswith.(adata.var_names, "RPS") .| startswith.(adata.var_names, "RPL")
adata.var.hb = occursin.(r"^HB[^P]", adata.var_names)

Juscan.Pp.calculate_qc_metrics!(adata, qc_vars=["mt", "ribo", "hb"])
adata = adata[adata.obs[!, "pct_counts_mt"] .< 8, :]
adata = adata[adata.obs[!, "pct_counts_mt"] .> 0.5, :]

Juscan.Pl.violin(
  adata,
  ["pct_counts_mt", "n_genes_by_counts", "total_counts"];
  width=300,
  height=800,
  fill_alpha=1,
  savefig="/home/lihuax/Pictures/Juscan/qc_violin.png",
)

Juscan.Pl.scatter(
  adata,
  "total_counts",
  "n_genes_by_counts",
  color_key="pct_counts_mt",
  width=800,
  height=800,
  colormap_name="magma",
  savefig="/home/lihuax/Pictures/Juscan/qc_scatter.png",
)

\end{lstlisting}
\end{fancybox}

运行上述代码后,获得处理后的高质量细胞数据集。为了直观展示质量控制的效果,使用小提琴图用于展示线粒体比例、总转录本数量与基因数的分布情况,帮助我们观察数据的整体质量分布;而散点图则揭示了转录本总数与检测到的基因数之间的关系,并以线粒体比例着色,从而更好地识别潜在的异常细胞。

图~\ref{img:Juscan_qc} 所示即为本次质量控制过程中生成的 QC 图像。

\begin{figure}[htbp]
  \centering
  \subfigure{
    \begin{minipage}[b]{.4\linewidth}
       \centering
       \includegraphics[scale=0.3]{./img/juscan_qc_violin.png}
    \end{minipage}
  }
  \subfigure{
    \begin{minipage}[b]{.4\linewidth}
      \centering
      \includegraphics[scale=0.3]{./img/juscan_qc_scatter.png}
    \end{minipage}
  }
  \caption{Juscan.jl质量控制可视化结果}
  \label{img:Juscan_qc}
\end{figure}

\subsection{数据归一化与高变基因}

首先,使用 \code{Juscan.Pp.normalize_total!} 函数对每个细胞的转录本数进行归一化,使其总表达量一致(如 1000),从而消除不同细胞间测序深度的影响。随后,应用 \code{Juscan.Tl.logp1_transform!} 函数对归一化后的数据进行对数变换($log(x+1)log(x+1)$),以缓解表达量跨度过大的问题,使其更符合后续线性模型的假设。

\begin{fancybox}{Juscan.jl normalize and find hvg}
\addcontentsline{lot}{table}{代码~\thetcbcounter: Juscan.jl示例:数据归一化与高变基因}
\begin{lstlisting}
############### normalization #################
adata.layers["normalized"] = deepcopy(adata.X)
Juscan.Pp.normalize_total!(adata, target_sum=1000, layer="normalized")
Juscan.Tl.logp1_transform!(adata, layer="normalized", key_added="normalized_logp1")
adata.layers["normalized_logp1"] = Float64.(adata.layers["normalized_logp1"])

############### highly variable genes #################
Juscan.Tl.highly_variable_genes!(adata, n_top_genes=2000, layer="normalized_logp1")
Juscan.Pl.hvg_scatter(adata, savefig="/home/lihuax/Pictures/Juscan/hvg_scatter.png")
\end{lstlisting}
\end{fancybox}

在归一化的基础上,使用 \code{Juscan.Tl.highly_variable_genes!} 方法从全基因集中筛选出表达波动性最强的 2000 个基因。这些基因包含了最丰富的生物学信号,是降维与聚类分析的核心输入。

图~\ref{img:hvg} 展示了PCA肘部图和高变基因的分布图,肘部图用于后续分析中选择pca维度数字而hvg图中显示了高变基因的分布,其中深色点为筛选出的高变基因。

\begin{figure}[htbp]
  \centering
  \subfigure{
    \begin{minipage}[b]{.27\linewidth}
       \centering
       \includegraphics[width=\textwidth]{./img/juscan_variance_ratio.png}
    \end{minipage}
  }
  \subfigure{
    \begin{minipage}[b]{.7\linewidth}
      \centering
      % \includegraphics[scale=0.4]{./img/juscan_qc_scatter.png}
        \includegraphics[width=\textwidth]{img/juscan_hvg_scatter.png}
    \end{minipage}
  }
  \caption{Juscan.jl归一化和高变基因可视化结果}
  \label{img:hvg}
\end{figure}

\subsection{降维与聚类}

在完成归一化与高变基因筛选之后,进行高维数据的压缩与结构发现。首先,通过 \code{Juscan.Tl.subset_to_hvg!} 筛选前 1000 个高变基因,以提取最具代表性的表达信号。随后,我们使用 \code{Juscan.Tl.pca!} 执行主成分分析(PCA),将原始维度浓缩为若干主成分,同时保存方差贡献比,便于选择合适的维度数进行后续分析。

紧接着,我们通过社区发现算法对 PCA 空间中的细胞进行聚类,以发现细胞的潜在亚群结构。通过 \code{Juscan.Tl.umap!} 函数进一步将 PCA 结果降维至二维空间,用于可视化聚类结果。

\begin{fancybox}{Juscan.jl dimensionality reduction and clustering}
\addcontentsline{lot}{table}{代码~\thetcbcounter: Juscan.jl示例:数据降维与聚类}
\begin{lstlisting}
############### dimensionality reduction #################
Juscan.Tl.subset_to_hvg!(adata; layer="normalized_logp2", n_top_genes=1000)

Juscan.Tl.pca!(adata; layer="normalized_logp1", key_added="pca", n_pcs=50)
Juscan.Pl.plot_variance_ratio(adata, savefig="/home/lihuax/Pictures/Juscan/variance_ratio.png")

############### clustering #################
Juscan.Tl.clustering!(adata, reduction="pca", use_pca=10, resolution=0.6)
Juscan.Tl.umap!(
  adata;
  key_added="umap",
  use_pca="pca",
  n_pcs=10,
  min_dist=0.5,
  n_neighbors=30,
)
Juscan.Pl.plot_umap(
  adata,
  color_by="clusters_latest",
  key="umap",
  savefig="/home/lihuax/Pictures/Juscan/clusters.png",
)
\end{lstlisting}
\end{fancybox}

图~\ref{img:clusters umap plot} 展示了 UMAP 嵌入空间中的聚类结果。每个点代表一个细胞,颜色区分不同的聚类标签。聚类结果的清晰分离,提示着数据中存在明显的生物学异质性,可能对应于不同类型的免疫细胞或功能状态。通过这样的降维可视化图,我们不仅能够观察整体结构,还能识别边缘群体与潜在的亚群细胞,为后续的注释与生物学解释奠定坚实基础。

\begin{figure}[htbp]
  \centering
  \includegraphics[width=0.7\textwidth]{img/juscan_clusters.png}
  \caption{Juscan.jl聚类结果umap可视化}
  \label{img:clusters umap plot}
\end{figure}

\section{性能评估}

为了全面评估不同语言实现对单细胞 RNA 测序数据处理流程的效率与性能差异,本文基于 PBMC 3k 数据集,分别使用 Julia、Python 与 R 三种主流语言构建了统一流程的 benchmark 测试系统。三种实现版本在逻辑结构、分析步骤及参数设置上保持一致,涵盖典型的单细胞分析流程,包括质量控制、归一化处理、高变基因选择、主成分分析(PCA)、聚类分析与 UMAP 可视化。

其中,Julia 的测试基于\code{BenchmarkTools.jl} 库, python和R分别使用\code{time}模块和\code{system.time}方法自行构建性能测试函数。
在每个步骤中,实验统一采用 100 次重复采样方式对每个函数进行性能测量,并最终输出时间与内存占用的统计汇总结果(CSV 格式)。

详细的测试代码请见附件。

\subsection{Seurat对比juscan.jl评估结果}

如下图\ref{img:r_vs_julia}是对比 Seurat 和 juscan.jl 的性能评估结果。

由于Seurat和Juscan的逻辑差异比较大,fiter和计算质量控制矩阵步骤没有被单独封装为函数,因此这里并没比较这些步骤的效率。

而对于已经比较的步骤,Juscan的效率明显优于Seurat,尤其在归一化,降维和聚类方面,Juscan的性能优势更为明显。

\begin{figure}[htbp]
  \centering
  \includegraphics[width=0.7\textwidth]{img/r_vs_julia.png}
  \caption{Juscan.jl与Seurat性能对比}
  \label{img:r_vs_julia}
\end{figure}

\subsection{scanpy对比juscan.jl评估结果}

如下图\ref{img:python_vs_julia}是对比 scanpy 和 juscan.jl 的性能评估结果。

可以明显看到,python在数据读取、归一化、PCA 和 UMAP 等步骤上均表现出显著的性能优势,尤其在降维方面。
另外Juscan的性能表现并不稳定,在不同的测试中,性能波动较大,可能与 Julia 的 JIT 编译机制有关。

总之,在归一化和降维,聚类等步骤上,Python 的性能明显优于 Julia,Juscan.jl任有恒大的进步空间。

\begin{figure}[htbp]
  \centering
  \includegraphics[width=0.8\textwidth]{img/python_vs_julia.png}
  \caption{Juscan.jl与Scanpy性能对比}
  \label{img:python_vs_julia}
\end{figure}



\chapter{讨论}
\section{组织蛋白酶同源基因的筛选}

为了深入探讨组织蛋白酶家族在秀丽隐杆线虫体内屏障衰老过程中的作用,我们首先 从线虫基因组数据库 WormBase 中系统性地筛选了组织蛋白酶家族的同源基因。

通过基因序列比对和功能注释分析,我们共筛选出了 40 种组织蛋白酶同源物(表1)。 这些基因在线虫体内的表达位置和表达水平各不相同,涵盖了从表皮到肠道等多个组织器  官。全面展示了组织蛋白酶家族在秀丽隐杆线虫中的同源基因名称及其表达位置,为研究 这些基因在衰老过程中的功能提供了重要的基础数据。在这 40 个基因中,有 7 个基因属  于组织蛋白酶 A 家族;10 个基因属于组织蛋白酶 B 家族;10 个基因属于组织蛋白酶 E 家  族;少量基因属于其他组织蛋白酶家族。且这些不同的基因表达的位置也各不相同,这些部 位主要包括神经、咽部、肠道、体壁肌等。

\begin{longtable}{lllp{7cm}}
  \caption{在线虫体内筛选出的 40 种同源物} \\
  \toprule
  \textbf{name} & \textbf{gene name} & \textbf{orthology} & \textbf{express in} \\
  \midrule
  \endfirsthead

  \multicolumn{4}{l}{\textit{续表:在线虫体内筛选出的 40 种同源物}} \\
  \toprule
  \textbf{name} & \textbf{gene name} & \textbf{orthology} & \textbf{express in} \\
  \midrule
  \endhead

  \bottomrule
  \multicolumn{4}{r}{\textit{表格接下页}} \\
  \endfoot

  \bottomrule
  \endlastfoot

K10C2.1 & ctsa-2 & CTSA & head mesodermal cell, intestine \\
Y40D12A.2 & drd-8, ctsa-2 & CTSA & head mesodermal cell, intestine, OLL, PVD neurons \\
K10B2.2 & ctsa-1 & CTSA & intestine, OLL, PVD neurons \\
C08H9.1 & ctsa-3.2 & CTSA & dopaminergic neuron \\
F32A5.3 & ctsa-3.1 & CTSA & head mesodermal cell, intestine \\
F41C3.5 & ctsa-1.1 & CTSA & BWM, germ line, gonad, head neurons, hypodermis, intestine, muscle cell, pharynx, reproductive system \\
F13D12.6 & ctsa-1.2 & CTSA & OLL PVD neurons, intestine, pharyngeal muscle cell \\
C52E4.1 & cpr-1 & CTSB & AVA DA OLL PVD SAB I5 neurons, head mesodermal cell, hypodermis, intestine, muscle \\
T10H4.12 & cpr-3 & CTSB & OLL PVD neurons, intestine, pharynx, rectal gland cell \\
F57F5.1 & cpr-9 & CTSB & DA OLL I5 PVD SAB neurons, intestine, head mesodermal cell \\
F44C4.3 & cpr-4 & CTSB & OLL PVD neurons, intestine, coelomocyte \\
W07B8.1 & cpr-8 & CTSB & intestine \\
F32H5.1 & \textbackslash & CTSB & intestine \\
C25B8.3 & cpr-6 & CTSB & intestine, pharyngeal muscle cell \\
W07B8.4 & \textbackslash & CTSB & intestine \\
F36D3.9 & cpr-2 & CTSB & OLL PLM PVD neurons, coelomocyte, intestine \\
W07B8.5 & cpr-5 & CTSB & OLL, PVD, intestine \\
Y113G7B.15 & \textbackslash & CTSC, CTSK, CTSS & DTC, gonad \\
Y51A2D.1 & \textbackslash & CTSC, CTSK, CTSS & AVK neurons \\
Y51A2D.8 & \textbackslash & CTSC, CTSW, CTSS & germ line, male-specific germ \\
R12H7.2 & asp-4 & CTSD & germ line, intestine, muscle cell \\
ZK384.3 & asp-18 & CTSE & head neurons, intestine, nervous system, pharynx \\
F21F8.3 & asp-5 & CTSE & OLL, PVD neurons, head mesodermal cell, intestine, pharyngeal muscle cell, germ line \\
Y39B6A.24 & asp-17 & CTSE & DA neurons, VA neurons, dopaminergic neurons, intestine \\
Y39B6A.20 & asp-1 & CTSE & head mesodermal cell, intestine, pharyngeal muscle cell \\
F59D6.3 & asp-8 & CTSE & Cephalic sheath, head mesodermal cell, intestine \\
K10C2.3 & asp-14 & CTSE & intestine, head mesodermal cell \\
F21F8.4 & asp-12 & CTSE & intestine \\
F59D6.2 & asp-7 & CTSE & Cephalic sheath \\
Y39B6A.23 & asp-16 & CTSE & \textbackslash \\
Y39B6A.22 & asp-15 & CTSE & \textbackslash \\
Y40H7A.10 & \textbackslash & CTSF & intestine \\
F41E6.6 & tag-196 & CTSF & body wall muscle, coelomocyte, intestine, head neurons, pharynx, tail neurons, vulva, VNC \\
R09F10.1 & \textbackslash & CTSF & intestine, pharyngeal muscle cell \\
K02E7.10 & \textbackslash & CTSK & hypodermis, intestine, somatic gonad precursor \\
Y71H2AM.25 & \textbackslash & CTSK & \textbackslash \\
T03E6.7 & cpl-1 & CTSL, VTSV & neurons, intestine, pharyngeal gland cell, cuticle, eggshell, gonadal sheath cell, hypodermis, muscle cell, uterus, vulva \\
M04G12.2 & cpz-2 & CTSZ & OLL, PVD neurons, head mesodermal cell, intestine, pharyngeal muscle cell \\
F32B5.8 & cpz-1 & CTSZ & AFD neurons, gonad, hypodermis, cuticle, intestine, pharynx, vulva \\

\end{longtable}

这些基因的多样性和广泛表达不仅表明组织蛋白酶家族在秀丽隐杆线虫的多个组织中 发挥着重要作用,而且暗示了它们在不同组织中可能具有不同的功能和调控机制。我们进 一步的研究分析揭示了这些基因在不同组织中的表达水平存在显著差异。

接下来我们使用热图展示了不同基因在年轻(D1   和年老(D8  ) 秀丽隐杆线虫不  同组织中的表达水平(图\ref{fig:heatmap1})。热图中的每行代表基因,每列代表年轻和年老线虫组织 , 不同的颜色代表基因表达水平的高低,颜色越红表示基因表达水平越高,颜色越蓝代

表基因表达水平越低。通过观察热图,我们可以发现以下几点——

首先,基因在不同组织中的表达水平存在显著差异。例如,某些基因在神经元中的表 达水平较高;而另一些基因在肠道或咽部的表达水平较高。这表明不同的组织蛋白酶基因 在不同的线虫组织中具有不同的功能和调控机制。

其次,某些基因在特定组织中表现出明显的高表达。例如,部分基因在头部中胚层细 胞、体壁肌和头部神经元中表现出较高的表达水平,而在其他组织中的表达则相对较低。这 种表达模式可能与该部分基因在这些组织衰老过程中的特定功能相关。

还有,这些组织蛋白酶在老年线虫的各个神经中,有的增加有的减少,这表示组织蛋白酶家族的成员并非通过统一的基因表达调控机制来调控,并且可能发挥不同的作用。

此外,热图中还显示了某些基因在多个组织中普遍表达,但表达水平均相对较低。这可能表明这些基因在维持组织的基本功能方面具有基础性的作用,但在特定组织中不发挥主要功能。

总的来说,这张热图揭示了组织蛋白酶家族基因在秀丽隐杆线虫不同组织中的表达式,为研究这些基因在衰老过程中的功能提供了重要的数据。通过进一步分析这些基因的表达水平在衰老过程中的变化,我们可以更好地理解这些基因在组织屏障功能维持和衰老相关病理变化中的作用。

\begin{figure}[H]
    \centering
    \includegraphics[width=0.8\textwidth]{img/heatmap1.png}
    \caption{组织蛋白酶同源物在秀丽隐杆线虫不同组织中的表达热图}
    \label{fig:heatmap1}
\end{figure}

\section{线虫肠道屏障实验结果}

在筛选出 40 种组织蛋白酶同源物并分析了它们在秀丽隐杆线虫不同组织中的表达模 式后,我们进一步探讨了这些基因在衰老相关屏障功能退变中的作用。为了验证这些基因 是否参与了线虫肠道屏障功能的调控,我们设计了一系列 RNAi 实验,通过特异性敲低目 标基因的表达,观察其对线虫肠道屏障完整性的影响。接下来, 我们将详细阐述这些 RNAi 实验的结果,揭示这些基因在肠道屏障功能维持和衰老过程中的具体作用。

Smurf Assay 实验是一种评估肠道屏障功能的方法, 用于检测肠道通透性的变化。该实 验通过向线虫饲喂含有蓝色染料的细菌,观察染料在肠道中的扩散情况,可以直观地评估 线虫肠道屏障的完整性。若染料扩散至肠道外边,则说明肠道屏障被破坏。

在我们的研究中,我们利用 Smurf Assay 实验评估了通过 RNAi 敲低组织蛋白酶同源 物后线虫肠道屏障功能的变化。通过观察蓝色染料在体腔中的扩散情况,我们能够直观地 评估这些基因对肠道屏障功能的影响。接下来,我将详细描述这些实验结果,揭示这些基 因在肠道屏障功能维持中的具体作用。

图\ref{fig:smurf_result}(a)展示了 Smurf Assay 实验的结果示意图。左侧图中, luc2 是对照组, 即表达一 种在线虫中不存在的乱序序列,不对任何线虫基因进行敲除;cpr-3 为经过 RNAi 实验敲

低 cpr-3 基因后的线虫。通过观察蓝色染料的扩散情况, 可以发现当我们敲低 cpr-3 基因的 表达量后,Smurf 染料(蓝色)扩散至肠道外侧的比例明显降低,表明该基因在线虫衰老过程中起到一定破坏线虫肠道屏障的作用。

图\ref{fig:smurf_result}(b)为 Smurf Assay 实验的定量结果示意图。横坐标代表了不同的被敲低的组织蛋 白酶基因,纵坐标代表该基因被敲低后线虫出现肠漏的比例。柱状图的高低反映了肠漏比 例的大小,肠漏比例越低,柱子越矮,表明该基因表达的组织蛋白酶越有可能参与降解线 虫的胞间连接蛋白,破坏线虫肠道屏障。

通过这一实验结果,我们筛选出了多个能够显著影响线虫肠道屏障功能的组织蛋白酶, 这些基因在维持肠道屏障功能和衰老过程中都发挥着重要的作用。

\begin{figure}[htbp]
  \centering
  \subfigure[原始 Smurf 装配图]{
    \includegraphics[width=0.45\linewidth]{img/smurf_assy.png}
  }
  \hfill
  \subfigure[装配结果展示图]{
    \includegraphics[width=0.5\linewidth]{img/smurf_assy_result.png}
  }
  \caption{线虫肠道屏障的 smurf assy 实验及定量结果}
  \label{fig:smurf_result}
\end{figure}


为了验证这些基因是否参与了线虫表皮屏障功能的调控,我们同样设计了一系列 RNAi 实验,以观察目的基因对线虫表皮屏障完整性的影响。

\section{线虫表皮屏障实验结果}

为了清楚展示组织蛋白酶对表皮屏障的破坏作用,我们使用绿色荧光标记HMR-1的线 虫株。在 488nm 波长荧光的激发下我们能够通过荧光共聚焦显微镜观察到线虫HMR-1的  表达和分布,从而代表表皮屏障的状态。当绿色荧光信号连续且紧密时,就代表这个线虫 的表皮屏障较为完整;但当点状绿色荧光信号之间的缝隙过大时,就代表其表皮屏障受到  了破坏。基于此原理,我们的实验结果如下——

\begin{figure}[H]
    \centering
    \includegraphics[width=0.8\textwidth]{img/fluorescence.png}
    \caption{四十种同源物在线虫体不同组织中的表达水平}
    \label{fig:fluorescence}
\end{figure}

图\ref{fig:fluorescence}左侧展示了荧光显微镜下线虫表皮屏障的实验结果示意图。左侧图中, luc2 为对  照组, 即未进行 RNAi 实验的线虫;右侧为经过 RNAi 实验敲低目标基因 cpr-1 后的线虫。 通过观察线虫 HMR-1-GFP 的荧光信号分布,我们可以发现当敲低某些组织蛋白酶后,线  虫表皮屏障的 HMR-1 分布更加连续且密集,空隙率显著降低,表明这些基因在衰老过程  中对线虫表皮屏障的破坏具有重要作用。

图\ref{fig:fluorescence}右侧为 RNAi 实验的定量结果示意图。横坐标代表了不同的被敲低的组织蛋白酶 基因,纵轴是做实验的次数,不同的色块代表该基因被敲低后线虫表皮屏障空隙率降低的 统计显著性,颜色越红的色块统计显著性越大,表示该基因在被敲低后,它的表皮屏障空 隙率越低,这个基因也越有可能讲解HMR-1。通过这一热图,我们可以直观地看到哪些基因的敲低显著改善了线虫表皮屏障的完整性。

这一实验我们同样也筛选出了多个能够显著影响线虫表皮屏障功能的组织蛋白酶同源 物。

\section{实验结论汇总}

通过以上两个实验,我们分别筛选出了一些具有破坏线虫组织屏障功能的组织蛋白酶。 我们取这两个线虫组织屏障实验筛选出的几种基因的交集, 我们便可以得到以下几个基因: asp-1;cpl-1;cpr1;cpr-3;cpr-4;cpr-6;ctsa-1.1(图\ref{fig:heatmap2})。这些基因的发现为我们理解生物体在衰  老过程中组织屏障功能退变的分子机制提供了全新的视角。我们的研究表明,这些组织蛋  白酶通过降解细胞间的连接蛋白,破坏了线虫肠道和表皮屏障的完整性,在衰老过程中发  挥了关键作用。通过进一步研究这些基因的功能和调控机制,我们可以更深入地了解衰老  过程中组织屏障功能的衰退现象,并探索延缓这一过程的可能性。这些研究成果不仅为线  虫的衰老研究提供了宝贵的实验数据,也为人类衰老相关疾病的研究和治疗提供了新的思  路和方向。

\begin{figure}[H]
    \centering
    \includegraphics[width=0.8\textwidth]{img/heatmap2.png}
    \caption{筛选出的具有破坏细胞间连接蛋白的组织蛋白酶基因}
    \label{fig:heatmap2}
\end{figure}

\section{总结}

本研究以秀丽隐杆线虫为模式生物,系统性地探讨了组织蛋白酶家族在屏障衰老过程 中的作用与机制。通过基因组数据库筛选和功能注释分析, 我们鉴定了 40 种组织蛋白酶家 族的同源基因, 并分析了它们在秀丽隐杆线虫不同组织中的表达模式。实验结果表明,这 些基因在神经、咽部、肠道和体壁肌等组织中呈现差异性表达,且部分基因在特定组织中 表现出显著的高表达,暗示了其在组织屏障功能维持中的重要作用。

为了验证这些基因在衰老相关屏障功能退变中的作用, 我们设计了一系列 RNAi 实验, 通过特异性敲低目标基因的表达, 观察其对线虫肠道和表皮屏障完整性的影响。 Smurf As-    say 实验结果显示,敲低某些组织蛋白酶同源基因后,线虫肠道屏障的通透性显著降低, 表  明这些基因在肠道屏障功能的破坏中发挥了重要作用。荧光共聚焦显微镜观察和分析进一步 表明,当敲低这些基因后,线虫表皮屏障的 HMR-1 蛋白分布更加连续且密集,空隙率显著 降低,这些该现象也说明了组织蛋白酶同源基因在表皮屏障功能的破坏中同样具有重要作  用。综合实验结果, 我们筛选出了 7 个关键基因:asp-1、cpl-1、cpr-1、cpr-3、cpr-4、cpr-6    和 ctsa-1.1,本研究不仅验证了既往关于 CPR-6 蛋白的研究结论,还扩展了组织蛋白酶家  族在衰老过程中的功能谱系。通过揭示组织蛋白酶家族成员在维持组织稳态和应对环境压  力中的作用,本研究为理解衰老相关病理变化提供了新的视角。鉴于线虫 HMR-1 蛋白与  人类连接蛋白的同源性,后续研究可将这些靶点转化到哺乳动物模型中,为临床应用提供  直接依据。此外,本研究还为延缓组织屏障功能衰退、改善老年相关疾病提供了创新疗法  的可能性,未来研究人员有望结合基因编辑技术和 RNA 干扰疗法等,实现对衰老过程的精  准干预。


%=================================================================
%论文后部
\backmatter


%=======%
%引入参考文献文件
%=======%
\bibdatabase{bib/database}%bib文件名称 仅修改bib/ 后部分
\printbib
\nocite{*} %显示数据库中有的,但是正文没有引用的文献

\Appendix

\section{Juscan.jl源代码与文档}

源码见Github仓库: \href{https://github.com/zehua0417/Juscan.jl.git}{https://github.com/zehua0417/Juscan.jl.git}

其他更多使用细节见文档:\href{https://zehua0417.github.io/Juscan.jl/}{https://zehua0417.github.io/Juscan.jl/}

\section{性能检测}

\showcode{julia}{Juscan.jl速度检测}{../vs/src/sc.jl}
\showcode{R}{Seurat速度检测}{../vs/src/sc.R}
\showcode{python}{scanpy速度检测}{../vs/src/sc.py}

\Thanks

本论文的完成离不开多方的关心与支持,在此谨致以诚挚的感谢。

首先,衷心感谢我的导师赵伟老师在课题选题、研究设计与论文撰写过程中给予的悉心指导与宝贵建议。赵老师严谨治学的态度、敏锐的问题意识以及持续探索的精神,为本研究的顺利推进提供了重要保障。

感谢生命科学学院为我提供了良好的学习与科研环境,系统的课程设置与丰富的学术资源,为我的专业成长和科研训练打下了坚实基础。感谢各位授课教师在本科阶段的辛勤教学和耐心指导,使我得以不断夯实专业知识,拓展学术视野。

感谢我的家人长期以来给予的理解与支持,是他们无声的鼓励与陪伴,使我能够专注于学业,克服困难,坚定前行。

同时,感谢开源社区与相关软件项目开发者的无私贡献,特别是Muon.jl、Scanpy等项目,为本研究提供了重要的参考与技术支持。

最后,感谢在项目开发、测试与讨论过程中给予我帮助与建议的同学与朋友们。你们的支持使本项目不断完善,持续前进。

\Grade

\end{document}
