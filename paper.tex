% !TEX TS-program = xelatex
% !TEX encoding = UTF-8 Unicode

% \documentclass[AutoFakeBold]{LZUThesis}
\documentclass[AutoFakeBold]{LZUThesis}
  
  \usepackage{wasysym}
  \usepackage{enumitem}
  \usepackage[most]{tcolorbox}
  \usepackage{multirow}
  \usepackage{inputenc}
  \usepackage{tikz}
  \usepackage{bbding}
  \usetikzlibrary{arrows.meta, decorations.markings}
  \usepackage{hyperref}
  \usepackage[numbers,sort&compress]{natbib}
  \usepackage{pdfpages}
  % \newcommand{\upcite}[1]{\textsuperscript{\textsuperscript{\cite{#1}}}}
  \allowdisplaybreaks[4]
  \usepackage{pdfpages}
  \usepackage[many]{tcolorbox}
  \usepackage{setspace}
  % \setmonofont{MapleMono-NF-CN-Medium}
  \setmonofont{MapleMono-NF-CN-Regular}
  
  % \newtcolorbox[auto counter, number within=section, list inside=listoftables]{fancybox}[2][]{
  \newcommand{\supercite}[1]{\textsuperscript{\textsuperscript{\cite{#1}}}}
  \newtcolorbox[use counter=table]{fancybox}[2][]{
      title=Code~\thetcbcounter: #2,
      breakable, left=1cm,
      label={code:#2},
      % list entry=code~\thetcbcounter: #2
      % colback=yellow!10!white, colframe=red!50!black
  }
  
  
  
  
  
  
  
  
  \usepackage[dvipsnames]{xcolor}
  \usepackage{tikz}
  \usetikzlibrary{backgrounds}
  \usetikzlibrary{arrows,shapes}
  \usetikzlibrary{tikzmark}
  \usetikzlibrary{calc}
  
  \usepackage{amsmath}
  \usepackage{amsthm}
  \usepackage{amssymb}
  \usepackage{mathtools, nccmath}
  \usepackage{wrapfig}
  \usepackage{comment}
  \usepackage{subfigure}
  
  % To generate dummy text
  \usepackage{blindtext}
  
  
  %color
  %\usepackage[dvipsnames]{xcolor}
  % \usepackage{xcolor}
  
  
  %\usepackage[pdftex]{graphicx}
  \usepackage{graphicx}
  % declare the path(s) for graphic files
  %\graphicspath{{../Figures/}}
  
  % extensions so you won't have to specify these with
  % every instance of \includegraphics
  % \DeclareGraphicsExtensions{.pdf,.jpeg,.png}
  
  % for custom commands
  \usepackage{xspace}
  
  % table alignment
  \usepackage{array}
  \usepackage{ragged2e}
  \newcolumntype{P}[1]{>{\RaggedRight\hspace{0pt}}p{#1}}
  \newcolumntype{X}[1]{>{\RaggedRight\hspace*{0pt}}p{#1}}
  
  % color box
  \usepackage{tcolorbox}
  
  
  % for tikz
  \usepackage{tikz}
  %\usetikzlibrary{trees}
  \usetikzlibrary{arrows,shapes,positioning,shadows,trees,mindmap}
  % \usepackage{forest}
  \usepackage[edges]{forest}
  \usetikzlibrary{arrows.meta}
  \colorlet{linecol}{black!75}
  \usepackage{xkcdcolors} % xkcd colors
  
  
  % for colorful equation
  \usepackage{tikz}
  \usetikzlibrary{backgrounds}
  \usetikzlibrary{arrows,shapes}
  \usetikzlibrary{tikzmark}
  \usetikzlibrary{calc}
  % Commands for Highlighting text -- non tikz method
  \newcommand{\highlight}[2]{\colorbox{#1!17}{$\displaystyle #2$}}
  %\newcommand{\highlight}[2]{\colorbox{#1!17}{$#2$}}
  \newcommand{\highlightdark}[2]{\colorbox{#1!47}{$\displaystyle #2$}}
  
  % my custom colors for shading
  \colorlet{mhpurple}{Plum!80}
  
  
  % Commands for Highlighting text -- non tikz method
  \renewcommand{\highlight}[2]{\colorbox{#1!17}{#2}}
  \renewcommand{\highlightdark}[2]{\colorbox{#1!47}{#2}}
  
  % Some math definitions
  \newcommand{\lap}{\mathrm{Lap}}
  \newcommand{\pr}{\mathrm{Pr}}
  
  \newcommand{\Tset}{\mathcal{T}}
  \newcommand{\Dset}{\mathcal{D}}
  \newcommand{\Rbound}{\widetilde{\mathcal{R}}}
  
  
  
  
  \newcommand{\code}[1]{\lstinline|#1|}





\begin{document}

\title{{组织蛋白酶家族对于}{体内屏障衰老的作用与机制}}

\entitle{{Cathepsin Family: Mechanisms and Impact}{on Aging of Physiological Barriers}}

\author{张蔚华}
\advisor{金卫林}
\major{生物学}
\college{萃英学院}
\grade{2021级}



\maketitle



%==============================%
% ↓ ↓ ↓ 诚信说明页 授权说明书
%==============================%

% 1. 可以调整签字的宽度,现在是40
% 2. 去掉raisebox的相关注释(注意上下大括号对应),可以改变-5那个数字调整签名和横线的上下位置

% 你的签名,signature.pdf 改为你的签名文件名,
\mysignature{
    % \raisebox{-5pt}{
    \includegraphics[width=60pt]{img/my_signature.png}
    % }
}
% 你手写的日期,signature.pdf 改为你的手写的日期文件名
\mytime{2025年4月27日}
% \mytime{
%     % \raisebox{-5pt}{
%     \includegraphics[width=40pt]{img/time_signature.pdf}
%     % }
% }
% 老师的手写签名,signature.pdf 改为老师的手写签名文件名
\supervisorsignature{
    % \raisebox{-5pt}{
    \includegraphics[width=60pt]{img/teacher_signature.png}
    % }
}
% 老师手写的时间,signature.pdf 改为老师的手写的日期文件名
\teachertime{2025年4月30日}
%     % \raisebox{-5pSt}{
%     \includegraphics[width=40pt]{img/teacher_time_signature.pdf}
%     % }
% }
% 老师手写的成绩
% \recommendedgrade{
%     % \raisebox{-5pt}{
%     \includegraphics[width=40pt]{img/score.pdf}
%     % }
% }

\makestatement

%==============================%
% ↑ ↑ ↑ 诚信说明页 授权说明书
%==============================%


%=====%
%论文(设计)成绩:注意2007的模板要求,成绩页在最后,2021要求成绩页在摘要前面
%=====%

\supervisorcomment{导师评价}
\recommendedgrade{0}

% \committeecomment{优秀}

% \finalgrade{100}
% 上面这些注释掉可以去掉成绩、评语什么的



\frontmatter



%中文摘要
\ZhAbstract{
在生物体中,细胞间的连接不仅发挥着细胞间信息交流和物质运输的作用,还维持着  组织屏障的完整性。然而随着衰老, 细胞间的连接被逐渐破坏, 生物体产生生理功能紊乱, 目前这种现象的潜在分子机制还未完全阐明。既往研究表示,在线虫模型中,分泌型组织  蛋白酶家族 B 在线虫中的其中一个同源物 CPR-6 蛋白的表达水平呈现年龄依赖性上调, 且  通过降解表皮细胞的胞间连接蛋白破坏组织屏障完整性, 这一发现表明组织蛋白酶 B 家族  在衰老相关的屏障功能失调中发挥关键作用。然而,组织蛋白酶在线虫中的其他同源物是  否参与这一生物学过程仍有待解析。因此, 在本研究中, 我们以秀丽隐杆线虫为模式生物, 理论上,我们第一次深入探索了组织蛋白酶在线虫中的其他同源物,并研究它们在衰老过  程中的作用;实验中,我们利用 RNA 干扰技术,特异性敲低线虫中相应组织蛋白酶的表达, 结合组织屏障完整性相关的实验,观察对比在衰老过程中线虫实验组与对照组之间组织屏  障的受损程度。实验数据揭示了有关组织蛋白酶在衰老过程中对组织屏障的调控机制,并  为开发基于屏障功能维持的衰老干预策略提供了潜在的分子靶点。
}{
衰老;秀丽隐杆线虫;组织蛋白酶家族;组织屏障;胞间连接蛋白
}


%英文摘要
\EnAbstract{
In organisms,intercellular connections are crucial for cellular communication, material trans- port, and maintaining the integrity of tissue barriers.  However, during aging, the gradual break- down of these connections can lead to physiological dysfunction. The underlying molecular mech- anisms are not fully understood.Previous studies in a nematode model revealed that the expression of CPR-6, a homolog of the secreted tissue plasminogen activator family B, increases with age. It disrupts tissue barriers by degrading epithelial cell junction proteins, indicating a key role of this family in age-related barrier dysfunction. Yet, the involvement of other homologs of this family in this process remains unclear.In this study, using Caenorhabditis elegans as a model, we explored the role of other tissue plasminogen activator homologs in aging. Through RNA interference, we specifically knocked down their expression and assessed tissue barrier integrity to compare the damage between experimental and control groups during aging.  Our data uncover the regulatory mechanisms of tissue plasminogen activators on tissue barriers during aging and offer potential molecular targets foraging intervention strategies based on barrier function maintenance.
}{
Aging;Caenorhabditis elegans;Cathepsin family;Tissue barrier;Cell junction proteins
}

%生成目录
\customcontent

%文章主体
\mainmatter
%=================================================================

\chapter{引言}
在生物体中,细胞间的连接包括紧密连接、黏附连接和间隙连接, 这些连接共同构成了 生物组织屏障的基础,确保了细胞间的正常连接交流和组织内部环境的稳定\upcite{noauthor__nodate}。然而,随着 机体的衰老, 细胞连接被逐渐破坏,出现组织屏障功能障碍以及生理功能紊乱的现象\upcite{bloom_mechanisms_2023} 。 在秀丽隐杆线虫这一经典模式生物中,表皮屏障作为抵御外界环境的第一道防线,其结构 完整性主要依赖于细胞间连接蛋白 HMR-1 。HMR-1 蛋白在维持表皮组织的紧密连接和屏 障功能方面发挥着关键作用\upcite{klompstra_instructive_2015}。前期研究显示, 随着线虫的衰老,表皮屏障中的 HMR-1 蛋 白逐渐降解。更深入的分子机制研究表明, 组织蛋白酶 B 家族在线虫中的同源物 CPR-6 在 此过程中起到降解 HMR-1 的作用,破坏线虫屏障的完整性。同时, 在线虫中,组织蛋白酶 家族成员众多且功能重叠度高, 这不由得引起我们的猜想,组织蛋白酶家族的其他成员是 否在衰老过程中参与降解胞间连接蛋白 HMR-1,并且最终导致机体出现严重的生理功能  障碍。基于上述研究背景, 本研究以秀丽隐杆线虫为模式生物, 探究组织蛋白酶家族在线  虫中的其他同源物是否具有与 CPR-6 类似的调控功能,我们利用 RNA 干扰技术特异性敲  低这些基因,并结合线虫组织屏障检查的相关实验, 筛选出了一些和 CPR-6 具有相同作用 的同源基因。通过这一研究,我们首次系统阐明了组织蛋白酶家族调控组织屏障稳态的分 子机制。本研究的结果不仅完善了衰老相关屏障功能障碍的理论框架,更为开发靶向组织 蛋白酶家族的抗衰老策略提供了重要理论依据。通过本研究,我们有望找到新的干预靶点 , 延缓组织屏障功能的衰退,延长生物体的健康寿命,具有重要的科学意义。


\chapter{背景介绍}
\section{衰老与秀丽隐杆线虫}

衰老是生命活动中一种极为复杂的生物学现象,它贯穿生命的整个历程。从分子层面 的微观变化,到器官乃至整个系统的宏观转变,无一不彰显着这一过程的深远影响。在生 物体衰老的过程中,细胞功能会逐渐衰退,细胞的代谢活性也随之降低。与此同时, 氧化应 激会不断在体内积累\upcite{lopez-otin_hallmarks_2023}。此外, DNA 修复能力也会逐渐丧失, 这些变化相互交织、相互影 响,最终使得生物个体的生存能力逐渐降低、适应能力下降、生命走向衰退\upcite{noauthor__nodate-1}。近年来, 衰 老研究逐渐成为生物医学领域的热点,不仅因为衰老是许多慢性疾病的主要风险因素,而 且它是人类健康寿命延长的重要方向\upcite{olshansky_implausibility_2024}。为了深入探索衰老的分子机制,科学家们开发了 多种模型生物,其中秀丽隐杆线虫因其独特的优势成为研究衰老的重要工具\upcite{jeayeng_caenorhabditis_2024} (图\ref{fig:worm_aging})。

\begin{figure}[H]
    \centering
    \includegraphics[width=0.8\textwidth]{img/worm_aging.jpg}
    \caption{线虫结构以及线虫衰老过程中的生理变化:(a) 线虫基本的身体数据;(b) 线虫在衰老过程中的生理变化}
    \label{fig:worm_aging}
\end{figure}

秀丽隐杆线虫因其生命周期短、遗传背景简单、易于操作以及完全测序的基因组等特 点,已成为研究衰老和发育生物学的理想模型生物\upcite{jeayeng_caenorhabditis_2024}。在秀丽隐杆线虫的衰老研究中,成 虫期从其被孵化后的第 1 天开始算起,其生殖期通常处于第 3-4 天,而衰老的迹象则从线 虫的第 8 天左右开始显现\upcite{jeayeng_caenorhabditis_2024}(图\ref{fig:worm_aging})。因此在研究中, 第一天被视为线虫成虫期的起点, 第 八天被视为线虫发生明显的衰老的时间线。这种时间上的对应关系使得科学家能够通过对 线虫进行不同时间点的比较,研究衰老的动态过程。

\section{胞间连接及组织屏障}

胞间连接是细胞间进行通信和物质交换的重要途径。细胞间的三种连接方式为紧密连 接,粘附连接和间隙连接。其中,紧密连接发挥着渗透屏障作用,粘附连接负责固定、维 持组织中细胞间的相对位置,间隙连接则负责信号传递(图\ref{fig:cell_connection})。

\begin{figure}[H]
    \centering
    \includegraphics[width=0.8\textwidth]{img/three_cell_link_functions.jpg}
    \caption{细胞间连接的三种方式:紧密连接、粘附连接和间隙连接}
    \label{fig:cell_connection}
\end{figure}

在线虫中,胞间连接主要通过缝隙连接实现。缝隙连接是一种特殊的细胞间结构,允 许小分子在相邻细胞之间直接传递。这种连接在多种生理过程中起着关键作用,包括发育 调控、代谢协调和应激响应等\upcite{wang_structural_2024}。在线虫中, HMR-1 蛋白是一种重要的缝隙连接蛋白, 且 HMR-1 与哺乳动物的连接蛋白具有高度同源性。HMR-1 在线虫的多种组织中表达,包括 表皮、肠道和生殖器官等(图\ref{fig:HMR-1_protein})。研究表明, HMR-1 不仅参与了细胞间的直接通信, 还 在胚胎发育、组织屏障功能和环境应激反应中发挥重要作用\upcite{naturale_persistent_2023}。

\begin{figure}[H]
    \centering
    \includegraphics[width=0.8\textwidth]{img/HMR_worm.png}
    \caption{实验用绿色荧光蛋白标记 HMR-1 来显示线虫的表皮屏障示意图}
    \label{fig:HMR-1_protein}
\end{figure}

线虫的组织屏障主要包括表皮屏障和肠道屏障,这两种屏障在维持线虫的内环境稳定  和防御外界病原体侵害中起着关键作用(图\ref{fig:worm_barrier})。线虫的表皮屏障由外层的表皮细胞构成,

主要负责保护线虫免受外界机械损伤和病原体的侵害。表皮细胞通过紧密连接形成了一种 屏障结构,防止有害物质从外界进入线虫体内\upcite{zhang_structural_2015}。肠道屏障是线虫消化系统的重要组成部 分,主要由肠道上皮细胞构成。肠道屏障的主要功能是吸收食物中的营养物质, 同时防止 有害物质扩散进入线虫体内。线虫肠道上皮细胞通过紧密连接,发挥选择性通透性屏障的 作用,确保营养物质的吸收和有害物质的排除\upcite{tan_killing_1999}。

\begin{figure}[H]
    \centering
    \includegraphics[width=0.8\textwidth]{img/worm_blocks.png}
    \caption{线虫的组织屏障示意图}
    \label{fig:worm_barrier}
\end{figure}

胞间连接和组织屏障在功能上具有密切的联系。胞间连接的异常状态会导致组织屏障 的功能发生障碍, 进而影响线虫的整体生理状态\upcite{wang_structural_2024}。因此, 通过对线虫 HMR-1 蛋白和组织 屏障的研究,可以为开发新的临床治疗策略提供理论依据。

\section{组织蛋白酶}

\begin{figure}[H]
    \centering
    \includegraphics[width=0.5\textwidth]{img/protease.jpg}
    \caption{组织蛋白酶破坏细胞间连接示意图}
    \label{fig:protease}
\end{figure}

组织蛋白酶是一类能够特异性分解细胞外基质成分的蛋白酶,广泛参与组织修复、细胞迁移、胚胎发育和疾病发生\upcite{jakos_cysteine_2019} 等过程\upcite{wang_cathepsins_2023}。在线虫中, 组织蛋白酶在细胞屏障的维持、胞 间连接的调控以及生理和病理过程中都具有重要的功能\upcite{teuscher_longevity_2024}。组织蛋白酶可能会通过调节细 胞外基质的成分,间接影响胞间连接的结构和功能。然而,组织蛋白酶的过度表达可能导 致细胞连接的稳定性失调,破坏细胞间的通信以及组织屏障(图\ref{fig:protease})。

\section{研究不同种类之间同源物的意义}

在生物学种系发生理论中,若两个或多个结构具有相同的祖先,则称它们同源。这里 相同的祖先既可以指演化意义上的祖先, 即两个结构由一个共同的祖先演化而来, 比如,蝙 蝠的翅膀与人类的手臂是同源的;也可以指发育意义上的祖先, 即两个结构由胚胎时期的 同一组织发育而来,例如,人类女性的卵巢与男性的睾丸同源。总而言之,若两个或多个 基因、蛋白、结构、组织或器官等具有相同的祖先,则称它们互为同源物。

在生物学研究中, 同源物的研究具有多方面的重要意义。比如,科学家可以通过比较 同源基因或者蛋白质,确定物种之间的亲缘关系,构建进化树,预测基因功能以及进行基 因注释等。而通过研究不同物种之间的同源物,尤其是人类在不同模式生物中的同源物对 于开发针对人类不同疾病的治疗方法等极为关键。这种方法不仅高效有用,而且能够最大 程度上降低生物实验的成本,对于科研人员具有重要的科研价值和实际应用意义。

\section{RNAi实验在线虫中的应用}

RNAi干扰是一种分子生物学上由双链RNA诱发的基因沉默现象,其机制是通过阻碍  特定基因的转录或翻译来抑制基因表达。当细胞中导入与内源性mRNA编码区同源的双链 RNA时,该mRNA会发生降解从而导致基因表达沉默\upcite{mello_revealing_2004}。
	
在线虫中,RNAi 实验应用广泛\upcite{tabara_rde-1_1999}。它不但能够研究基因的功能,还能够确定信号通路 中不同基因的上下游关系,帮助我们理解细胞里的信息传递网络。线虫对 RNAi 反应敏感 高效,只需要通过喂食,就能轻松把 dsRNA送入线虫体内,引发 RNAi 反应。RNAi 实验在 线虫中的意义重大。它不仅推动了研究人员对线虫基因功能的认识,还为理解人类基因功 能和疾病机制提供了线索。而在我们的实验中之所以选择 RNAi 来敲低目的基因,是因为 我们的实验需要筛选目的基因,该试验便于进行大规模的基因筛选。

\section{小结}

本章节从衰老这一生命活动的复杂生物学现象出发, 阐述了秀丽隐杆线虫作为研究衰 老的重要模式生物的优势;进而深入探讨了胞间连接里紧密连接、粘附连接和间隙连接的 功能,且着重介绍了线虫缝隙连接及 HMR-1 蛋白的关键作用;接着又介绍了线虫组织屏障的构成与功能,以及胞间连接与组织屏障的密切联系;随后聚焦于组织蛋白酶, 阐述其 广泛参与多种生理病理过程的现象及其对胞间连接和组织屏障的潜在影响;最后介绍RNAi 实验原理及在线虫中的应用优势, 为后续基于 RNAi 的基因功能研究实验介绍奠定了 基础。


\chapter{实验流程与方法}
%%% 实验与评估

\section{实验数据集介绍}

本研究使用的单细胞转录组数据集为 10x Genomics 公司公开发布的经典示例数据集——PBMC 3k 数据集。该数据集包含约 2,700 个外周血单核细胞(Peripheral Blood Mononuclear Cells, PBMC),来自一位健康成年供体。PBMC 包括 T 细胞、B 细胞、自然杀伤细胞和单核细胞等多种免疫相关细胞类型,常作为免疫学研究中的模型系统。

PBMC 3k 数据集采用 10x Genomics Chromium 单细胞 3' RNA 测序平台进行文库构建,并使用 Illumina NextSeq 500 高通量测序仪进行测序。每个细胞平均测序深度约为 69,000 个 reads。原始数据使用 10x Genomics 官方分析工具 Cell Ranger(版本 1.1.0)进行预处理,包括细胞条形码识别、UMI 去重及基因表达矩阵生成。数据结果以稀疏矩阵形式存储,包含每个细胞在各个基因上的表达计数。

本数据集可以在\href{https://cf.10xgenomics.com/samples/cell/pbmc3k/pbmc3k_filtered_gene_bc_matrices.tar.gz}{10XGenomics官网}获取,并可以使用SeuratDisk,scDior等工具转换为h5ad文件。

由于数据来源权威、质量可靠、样本复杂性适中,PBMC 3k 数据集已广泛应用于 Seurat、Scanpy 等单细胞分析工具的功能验证与流程教学中,是验证聚类、降维、批次效应处理等算法性能的标准数据集之一。


\section{实验设计与流程}

本文基于 PBMC 3k 数据集,参考官方流程,实现了单细胞 RNA 测序数据分析的完整流程。以下是使用 Juscan.jl 库进行数据分析的示例:

\subsection{数据读取与质量控制}

Juscan.jl 库使用 Muon.jl 库作为数据存储的基础,支持读取 h5ad 格式的数据文件。数据读取后,使用\code{Juscan.Pp.filter_cells!} 和 \code{Juscan.Pp.filter_genes!} 函数进行细胞和基因的过滤,去除低质量细胞和低表达基因。接着,计算线粒体基因、核糖体基因和血红蛋白基因的表达比例,并根据这些指标进行进一步的质量控制。

\begin{fancybox}{Juscan.jl demo data load and quality control}
\addcontentsline{lot}{table}{代码~\thetcbcounter: Juscan.jl示例:数据读取与质量控制}
\begin{lstlisting}
using Juscan, Muon
using DataFrames, LinearAlgebra, SparseArrays

adata = Juscan.readh5ad("./data/pbmc_3k.h5ad")

############## quality control #################

Juscan.Pp.filter_cells!(adata, min_genes=200)
Juscan.Pp.filter_genes!(adata, min_cells=3)
Juscan.Pp.filter_cells!(adata, max_genes=2300)
Juscan.Pp.filter_cells!(adata, max_counts=10000)

adata.var.mt = startswith.(adata.var_names, "MT-")
adata.var.ribo = startswith.(adata.var_names, "RPS") .| startswith.(adata.var_names, "RPL")
adata.var.hb = occursin.(r"^HB[^P]", adata.var_names)

Juscan.Pp.calculate_qc_metrics!(adata, qc_vars=["mt", "ribo", "hb"])
adata = adata[adata.obs[!, "pct_counts_mt"] .< 8, :]
adata = adata[adata.obs[!, "pct_counts_mt"] .> 0.5, :]

Juscan.Pl.violin(
  adata,
  ["pct_counts_mt", "n_genes_by_counts", "total_counts"];
  width=300,
  height=800,
  fill_alpha=1,
  savefig="/home/lihuax/Pictures/Juscan/qc_violin.png",
)

Juscan.Pl.scatter(
  adata,
  "total_counts",
  "n_genes_by_counts",
  color_key="pct_counts_mt",
  width=800,
  height=800,
  colormap_name="magma",
  savefig="/home/lihuax/Pictures/Juscan/qc_scatter.png",
)

\end{lstlisting}
\end{fancybox}

运行上述代码后,获得处理后的高质量细胞数据集。为了直观展示质量控制的效果,使用小提琴图用于展示线粒体比例、总转录本数量与基因数的分布情况,帮助我们观察数据的整体质量分布;而散点图则揭示了转录本总数与检测到的基因数之间的关系,并以线粒体比例着色,从而更好地识别潜在的异常细胞。

图~\ref{img:Juscan_qc} 所示即为本次质量控制过程中生成的 QC 图像。

\begin{figure}[htbp]
  \centering
  \subfigure{
    \begin{minipage}[b]{.4\linewidth}
       \centering
       \includegraphics[scale=0.3]{./img/juscan_qc_violin.png}
    \end{minipage}
  }
  \subfigure{
    \begin{minipage}[b]{.4\linewidth}
      \centering
      \includegraphics[scale=0.3]{./img/juscan_qc_scatter.png}
    \end{minipage}
  }
  \caption{Juscan.jl质量控制可视化结果}
  \label{img:Juscan_qc}
\end{figure}

\subsection{数据归一化与高变基因}

首先,使用 \code{Juscan.Pp.normalize_total!} 函数对每个细胞的转录本数进行归一化,使其总表达量一致(如 1000),从而消除不同细胞间测序深度的影响。随后,应用 \code{Juscan.Tl.logp1_transform!} 函数对归一化后的数据进行对数变换($log(x+1)log(x+1)$),以缓解表达量跨度过大的问题,使其更符合后续线性模型的假设。

\begin{fancybox}{Juscan.jl normalize and find hvg}
\addcontentsline{lot}{table}{代码~\thetcbcounter: Juscan.jl示例:数据归一化与高变基因}
\begin{lstlisting}
############### normalization #################
adata.layers["normalized"] = deepcopy(adata.X)
Juscan.Pp.normalize_total!(adata, target_sum=1000, layer="normalized")
Juscan.Tl.logp1_transform!(adata, layer="normalized", key_added="normalized_logp1")
adata.layers["normalized_logp1"] = Float64.(adata.layers["normalized_logp1"])

############### highly variable genes #################
Juscan.Tl.highly_variable_genes!(adata, n_top_genes=2000, layer="normalized_logp1")
Juscan.Pl.hvg_scatter(adata, savefig="/home/lihuax/Pictures/Juscan/hvg_scatter.png")
\end{lstlisting}
\end{fancybox}

在归一化的基础上,使用 \code{Juscan.Tl.highly_variable_genes!} 方法从全基因集中筛选出表达波动性最强的 2000 个基因。这些基因包含了最丰富的生物学信号,是降维与聚类分析的核心输入。

图~\ref{img:hvg} 展示了PCA肘部图和高变基因的分布图,肘部图用于后续分析中选择pca维度数字而hvg图中显示了高变基因的分布,其中深色点为筛选出的高变基因。

\begin{figure}[htbp]
  \centering
  \subfigure{
    \begin{minipage}[b]{.27\linewidth}
       \centering
       \includegraphics[width=\textwidth]{./img/juscan_variance_ratio.png}
    \end{minipage}
  }
  \subfigure{
    \begin{minipage}[b]{.7\linewidth}
      \centering
      % \includegraphics[scale=0.4]{./img/juscan_qc_scatter.png}
        \includegraphics[width=\textwidth]{img/juscan_hvg_scatter.png}
    \end{minipage}
  }
  \caption{Juscan.jl归一化和高变基因可视化结果}
  \label{img:hvg}
\end{figure}

\subsection{降维与聚类}

在完成归一化与高变基因筛选之后,进行高维数据的压缩与结构发现。首先,通过 \code{Juscan.Tl.subset_to_hvg!} 筛选前 1000 个高变基因,以提取最具代表性的表达信号。随后,我们使用 \code{Juscan.Tl.pca!} 执行主成分分析(PCA),将原始维度浓缩为若干主成分,同时保存方差贡献比,便于选择合适的维度数进行后续分析。

紧接着,我们通过社区发现算法对 PCA 空间中的细胞进行聚类,以发现细胞的潜在亚群结构。通过 \code{Juscan.Tl.umap!} 函数进一步将 PCA 结果降维至二维空间,用于可视化聚类结果。

\begin{fancybox}{Juscan.jl dimensionality reduction and clustering}
\addcontentsline{lot}{table}{代码~\thetcbcounter: Juscan.jl示例:数据降维与聚类}
\begin{lstlisting}
############### dimensionality reduction #################
Juscan.Tl.subset_to_hvg!(adata; layer="normalized_logp2", n_top_genes=1000)

Juscan.Tl.pca!(adata; layer="normalized_logp1", key_added="pca", n_pcs=50)
Juscan.Pl.plot_variance_ratio(adata, savefig="/home/lihuax/Pictures/Juscan/variance_ratio.png")

############### clustering #################
Juscan.Tl.clustering!(adata, reduction="pca", use_pca=10, resolution=0.6)
Juscan.Tl.umap!(
  adata;
  key_added="umap",
  use_pca="pca",
  n_pcs=10,
  min_dist=0.5,
  n_neighbors=30,
)
Juscan.Pl.plot_umap(
  adata,
  color_by="clusters_latest",
  key="umap",
  savefig="/home/lihuax/Pictures/Juscan/clusters.png",
)
\end{lstlisting}
\end{fancybox}

图~\ref{img:clusters umap plot} 展示了 UMAP 嵌入空间中的聚类结果。每个点代表一个细胞,颜色区分不同的聚类标签。聚类结果的清晰分离,提示着数据中存在明显的生物学异质性,可能对应于不同类型的免疫细胞或功能状态。通过这样的降维可视化图,我们不仅能够观察整体结构,还能识别边缘群体与潜在的亚群细胞,为后续的注释与生物学解释奠定坚实基础。

\begin{figure}[htbp]
  \centering
  \includegraphics[width=0.7\textwidth]{img/juscan_clusters.png}
  \caption{Juscan.jl聚类结果umap可视化}
  \label{img:clusters umap plot}
\end{figure}

\section{性能评估}

为了全面评估不同语言实现对单细胞 RNA 测序数据处理流程的效率与性能差异,本文基于 PBMC 3k 数据集,分别使用 Julia、Python 与 R 三种主流语言构建了统一流程的 benchmark 测试系统。三种实现版本在逻辑结构、分析步骤及参数设置上保持一致,涵盖典型的单细胞分析流程,包括质量控制、归一化处理、高变基因选择、主成分分析(PCA)、聚类分析与 UMAP 可视化。

其中,Julia 的测试基于\code{BenchmarkTools.jl} 库, python和R分别使用\code{time}模块和\code{system.time}方法自行构建性能测试函数。
在每个步骤中,实验统一采用 100 次重复采样方式对每个函数进行性能测量,并最终输出时间与内存占用的统计汇总结果(CSV 格式)。

详细的测试代码请见附件。

\subsection{Seurat对比juscan.jl评估结果}

如下图\ref{img:r_vs_julia}是对比 Seurat 和 juscan.jl 的性能评估结果。

由于Seurat和Juscan的逻辑差异比较大,fiter和计算质量控制矩阵步骤没有被单独封装为函数,因此这里并没比较这些步骤的效率。

而对于已经比较的步骤,Juscan的效率明显优于Seurat,尤其在归一化,降维和聚类方面,Juscan的性能优势更为明显。

\begin{figure}[htbp]
  \centering
  \includegraphics[width=0.7\textwidth]{img/r_vs_julia.png}
  \caption{Juscan.jl与Seurat性能对比}
  \label{img:r_vs_julia}
\end{figure}

\subsection{scanpy对比juscan.jl评估结果}

如下图\ref{img:python_vs_julia}是对比 scanpy 和 juscan.jl 的性能评估结果。

可以明显看到,python在数据读取、归一化、PCA 和 UMAP 等步骤上均表现出显著的性能优势,尤其在降维方面。
另外Juscan的性能表现并不稳定,在不同的测试中,性能波动较大,可能与 Julia 的 JIT 编译机制有关。

总之,在归一化和降维,聚类等步骤上,Python 的性能明显优于 Julia,Juscan.jl任有恒大的进步空间。

\begin{figure}[htbp]
  \centering
  \includegraphics[width=0.8\textwidth]{img/python_vs_julia.png}
  \caption{Juscan.jl与Scanpy性能对比}
  \label{img:python_vs_julia}
\end{figure}



\chapter{实验结果}
\section{组织蛋白酶同源基因的筛选}

为了深入探讨组织蛋白酶家族在秀丽隐杆线虫体内屏障衰老过程中的作用,我们首先 从线虫基因组数据库 WormBase 中系统性地筛选了组织蛋白酶家族的同源基因。

通过基因序列比对和功能注释分析,我们共筛选出了 40 种组织蛋白酶同源物(表1)。 这些基因在线虫体内的表达位置和表达水平各不相同,涵盖了从表皮到肠道等多个组织器  官。全面展示了组织蛋白酶家族在秀丽隐杆线虫中的同源基因名称及其表达位置,为研究 这些基因在衰老过程中的功能提供了重要的基础数据。在这 40 个基因中,有 7 个基因属  于组织蛋白酶 A 家族;10 个基因属于组织蛋白酶 B 家族;10 个基因属于组织蛋白酶 E 家  族;少量基因属于其他组织蛋白酶家族。且这些不同的基因表达的位置也各不相同,这些部 位主要包括神经、咽部、肠道、体壁肌等。

\begin{longtable}{lllp{7cm}}
  \caption{在线虫体内筛选出的 40 种同源物} \\
  \toprule
  \textbf{name} & \textbf{gene name} & \textbf{orthology} & \textbf{express in} \\
  \midrule
  \endfirsthead

  \multicolumn{4}{l}{\textit{续表:在线虫体内筛选出的 40 种同源物}} \\
  \toprule
  \textbf{name} & \textbf{gene name} & \textbf{orthology} & \textbf{express in} \\
  \midrule
  \endhead

  \bottomrule
  \multicolumn{4}{r}{\textit{表格接下页}} \\
  \endfoot

  \bottomrule
  \endlastfoot

K10C2.1 & ctsa-2 & CTSA & head mesodermal cell, intestine \\
Y40D12A.2 & drd-8, ctsa-2 & CTSA & head mesodermal cell, intestine, OLL, PVD neurons \\
K10B2.2 & ctsa-1 & CTSA & intestine, OLL, PVD neurons \\
C08H9.1 & ctsa-3.2 & CTSA & dopaminergic neuron \\
F32A5.3 & ctsa-3.1 & CTSA & head mesodermal cell, intestine \\
F41C3.5 & ctsa-1.1 & CTSA & BWM, germ line, gonad, head neurons, hypodermis, intestine, muscle cell, pharynx, reproductive system \\
F13D12.6 & ctsa-1.2 & CTSA & OLL PVD neurons, intestine, pharyngeal muscle cell \\
C52E4.1 & cpr-1 & CTSB & AVA DA OLL PVD SAB I5 neurons, head mesodermal cell, hypodermis, intestine, muscle \\
T10H4.12 & cpr-3 & CTSB & OLL PVD neurons, intestine, pharynx, rectal gland cell \\
F57F5.1 & cpr-9 & CTSB & DA OLL I5 PVD SAB neurons, intestine, head mesodermal cell \\
F44C4.3 & cpr-4 & CTSB & OLL PVD neurons, intestine, coelomocyte \\
W07B8.1 & cpr-8 & CTSB & intestine \\
F32H5.1 & \textbackslash & CTSB & intestine \\
C25B8.3 & cpr-6 & CTSB & intestine, pharyngeal muscle cell \\
W07B8.4 & \textbackslash & CTSB & intestine \\
F36D3.9 & cpr-2 & CTSB & OLL PLM PVD neurons, coelomocyte, intestine \\
W07B8.5 & cpr-5 & CTSB & OLL, PVD, intestine \\
Y113G7B.15 & \textbackslash & CTSC, CTSK, CTSS & DTC, gonad \\
Y51A2D.1 & \textbackslash & CTSC, CTSK, CTSS & AVK neurons \\
Y51A2D.8 & \textbackslash & CTSC, CTSW, CTSS & germ line, male-specific germ \\
R12H7.2 & asp-4 & CTSD & germ line, intestine, muscle cell \\
ZK384.3 & asp-18 & CTSE & head neurons, intestine, nervous system, pharynx \\
F21F8.3 & asp-5 & CTSE & OLL, PVD neurons, head mesodermal cell, intestine, pharyngeal muscle cell, germ line \\
Y39B6A.24 & asp-17 & CTSE & DA neurons, VA neurons, dopaminergic neurons, intestine \\
Y39B6A.20 & asp-1 & CTSE & head mesodermal cell, intestine, pharyngeal muscle cell \\
F59D6.3 & asp-8 & CTSE & Cephalic sheath, head mesodermal cell, intestine \\
K10C2.3 & asp-14 & CTSE & intestine, head mesodermal cell \\
F21F8.4 & asp-12 & CTSE & intestine \\
F59D6.2 & asp-7 & CTSE & Cephalic sheath \\
Y39B6A.23 & asp-16 & CTSE & \textbackslash \\
Y39B6A.22 & asp-15 & CTSE & \textbackslash \\
Y40H7A.10 & \textbackslash & CTSF & intestine \\
F41E6.6 & tag-196 & CTSF & body wall muscle, coelomocyte, intestine, head neurons, pharynx, tail neurons, vulva, VNC \\
R09F10.1 & \textbackslash & CTSF & intestine, pharyngeal muscle cell \\
K02E7.10 & \textbackslash & CTSK & hypodermis, intestine, somatic gonad precursor \\
Y71H2AM.25 & \textbackslash & CTSK & \textbackslash \\
T03E6.7 & cpl-1 & CTSL, VTSV & neurons, intestine, pharyngeal gland cell, cuticle, eggshell, gonadal sheath cell, hypodermis, muscle cell, uterus, vulva \\
M04G12.2 & cpz-2 & CTSZ & OLL, PVD neurons, head mesodermal cell, intestine, pharyngeal muscle cell \\
F32B5.8 & cpz-1 & CTSZ & AFD neurons, gonad, hypodermis, cuticle, intestine, pharynx, vulva \\

\end{longtable}

这些基因的多样性和广泛表达不仅表明组织蛋白酶家族在秀丽隐杆线虫的多个组织中 发挥着重要作用,而且暗示了它们在不同组织中可能具有不同的功能和调控机制。我们进 一步的研究分析揭示了这些基因在不同组织中的表达水平存在显著差异。

接下来我们使用热图展示了不同基因在年轻(D1   和年老(D8  ) 秀丽隐杆线虫不  同组织中的表达水平(图\ref{fig:heatmap1})。热图中的每行代表基因,每列代表年轻和年老线虫组织 , 不同的颜色代表基因表达水平的高低,颜色越红表示基因表达水平越高,颜色越蓝代

表基因表达水平越低。通过观察热图,我们可以发现以下几点——

首先,基因在不同组织中的表达水平存在显著差异。例如,某些基因在神经元中的表 达水平较高;而另一些基因在肠道或咽部的表达水平较高。这表明不同的组织蛋白酶基因 在不同的线虫组织中具有不同的功能和调控机制。

其次,某些基因在特定组织中表现出明显的高表达。例如,部分基因在头部中胚层细 胞、体壁肌和头部神经元中表现出较高的表达水平,而在其他组织中的表达则相对较低。这 种表达模式可能与该部分基因在这些组织衰老过程中的特定功能相关。

还有,这些组织蛋白酶在老年线虫的各个神经中,有的增加有的减少,这表示组织蛋白酶家族的成员并非通过统一的基因表达调控机制来调控,并且可能发挥不同的作用。

此外,热图中还显示了某些基因在多个组织中普遍表达,但表达水平均相对较低。这可能表明这些基因在维持组织的基本功能方面具有基础性的作用,但在特定组织中不发挥主要功能。

总的来说,这张热图揭示了组织蛋白酶家族基因在秀丽隐杆线虫不同组织中的表达式,为研究这些基因在衰老过程中的功能提供了重要的数据。通过进一步分析这些基因的表达水平在衰老过程中的变化,我们可以更好地理解这些基因在组织屏障功能维持和衰老相关病理变化中的作用。

\begin{figure}[H]
    \centering
    \includegraphics[width=0.8\textwidth]{img/heatmap1.png}
    \caption{组织蛋白酶同源物在秀丽隐杆线虫不同组织中的表达热图}
    \label{fig:heatmap1}
\end{figure}

\section{线虫肠道屏障实验结果}

在筛选出 40 种组织蛋白酶同源物并分析了它们在秀丽隐杆线虫不同组织中的表达模 式后,我们进一步探讨了这些基因在衰老相关屏障功能退变中的作用。为了验证这些基因 是否参与了线虫肠道屏障功能的调控,我们设计了一系列 RNAi 实验,通过特异性敲低目 标基因的表达,观察其对线虫肠道屏障完整性的影响。接下来, 我们将详细阐述这些 RNAi 实验的结果,揭示这些基因在肠道屏障功能维持和衰老过程中的具体作用。

Smurf Assay 实验是一种评估肠道屏障功能的方法, 用于检测肠道通透性的变化。该实 验通过向线虫饲喂含有蓝色染料的细菌,观察染料在肠道中的扩散情况,可以直观地评估 线虫肠道屏障的完整性。若染料扩散至肠道外边,则说明肠道屏障被破坏。

在我们的研究中,我们利用 Smurf Assay 实验评估了通过 RNAi 敲低组织蛋白酶同源 物后线虫肠道屏障功能的变化。通过观察蓝色染料在体腔中的扩散情况,我们能够直观地 评估这些基因对肠道屏障功能的影响。接下来,我将详细描述这些实验结果,揭示这些基 因在肠道屏障功能维持中的具体作用。

图\ref{fig:smurf_result}(a)展示了 Smurf Assay 实验的结果示意图。左侧图中, luc2 是对照组, 即表达一 种在线虫中不存在的乱序序列,不对任何线虫基因进行敲除;cpr-3 为经过 RNAi 实验敲

低 cpr-3 基因后的线虫。通过观察蓝色染料的扩散情况, 可以发现当我们敲低 cpr-3 基因的 表达量后,Smurf 染料(蓝色)扩散至肠道外侧的比例明显降低,表明该基因在线虫衰老过程中起到一定破坏线虫肠道屏障的作用。

图\ref{fig:smurf_result}(b)为 Smurf Assay 实验的定量结果示意图。横坐标代表了不同的被敲低的组织蛋 白酶基因,纵坐标代表该基因被敲低后线虫出现肠漏的比例。柱状图的高低反映了肠漏比 例的大小,肠漏比例越低,柱子越矮,表明该基因表达的组织蛋白酶越有可能参与降解线 虫的胞间连接蛋白,破坏线虫肠道屏障。

通过这一实验结果,我们筛选出了多个能够显著影响线虫肠道屏障功能的组织蛋白酶, 这些基因在维持肠道屏障功能和衰老过程中都发挥着重要的作用。

\begin{figure}[htbp]
  \centering
  \subfigure[原始 Smurf 装配图]{
    \includegraphics[width=0.45\linewidth]{img/smurf_assy.png}
  }
  \hfill
  \subfigure[装配结果展示图]{
    \includegraphics[width=0.5\linewidth]{img/smurf_assy_result.png}
  }
  \caption{线虫肠道屏障的 smurf assy 实验及定量结果}
  \label{fig:smurf_result}
\end{figure}


为了验证这些基因是否参与了线虫表皮屏障功能的调控,我们同样设计了一系列 RNAi 实验,以观察目的基因对线虫表皮屏障完整性的影响。

\section{线虫表皮屏障实验结果}

为了清楚展示组织蛋白酶对表皮屏障的破坏作用,我们使用绿色荧光标记HMR-1的线 虫株。在 488nm 波长荧光的激发下我们能够通过荧光共聚焦显微镜观察到线虫HMR-1的  表达和分布,从而代表表皮屏障的状态。当绿色荧光信号连续且紧密时,就代表这个线虫 的表皮屏障较为完整;但当点状绿色荧光信号之间的缝隙过大时,就代表其表皮屏障受到  了破坏。基于此原理,我们的实验结果如下——

\begin{figure}[H]
    \centering
    \includegraphics[width=0.8\textwidth]{img/fluorescence.png}
    \caption{四十种同源物在线虫体不同组织中的表达水平}
    \label{fig:fluorescence}
\end{figure}

图\ref{fig:fluorescence}左侧展示了荧光显微镜下线虫表皮屏障的实验结果示意图。左侧图中, luc2 为对  照组, 即未进行 RNAi 实验的线虫;右侧为经过 RNAi 实验敲低目标基因 cpr-1 后的线虫。 通过观察线虫 HMR-1-GFP 的荧光信号分布,我们可以发现当敲低某些组织蛋白酶后,线  虫表皮屏障的 HMR-1 分布更加连续且密集,空隙率显著降低,表明这些基因在衰老过程  中对线虫表皮屏障的破坏具有重要作用。

图\ref{fig:fluorescence}右侧为 RNAi 实验的定量结果示意图。横坐标代表了不同的被敲低的组织蛋白酶 基因,纵轴是做实验的次数,不同的色块代表该基因被敲低后线虫表皮屏障空隙率降低的 统计显著性,颜色越红的色块统计显著性越大,表示该基因在被敲低后,它的表皮屏障空 隙率越低,这个基因也越有可能讲解HMR-1。通过这一热图,我们可以直观地看到哪些基因的敲低显著改善了线虫表皮屏障的完整性。

这一实验我们同样也筛选出了多个能够显著影响线虫表皮屏障功能的组织蛋白酶同源 物。

\section{实验结论汇总}

通过以上两个实验,我们分别筛选出了一些具有破坏线虫组织屏障功能的组织蛋白酶。 我们取这两个线虫组织屏障实验筛选出的几种基因的交集, 我们便可以得到以下几个基因: asp-1;cpl-1;cpr1;cpr-3;cpr-4;cpr-6;ctsa-1.1(图\ref{fig:heatmap2})。这些基因的发现为我们理解生物体在衰  老过程中组织屏障功能退变的分子机制提供了全新的视角。我们的研究表明,这些组织蛋  白酶通过降解细胞间的连接蛋白,破坏了线虫肠道和表皮屏障的完整性,在衰老过程中发  挥了关键作用。通过进一步研究这些基因的功能和调控机制,我们可以更深入地了解衰老  过程中组织屏障功能的衰退现象,并探索延缓这一过程的可能性。这些研究成果不仅为线  虫的衰老研究提供了宝贵的实验数据,也为人类衰老相关疾病的研究和治疗提供了新的思  路和方向。

\begin{figure}[H]
    \centering
    \includegraphics[width=0.8\textwidth]{img/heatmap2.png}
    \caption{筛选出的具有破坏细胞间连接蛋白的组织蛋白酶基因}
    \label{fig:heatmap2}
\end{figure}

\section{总结}

本研究以秀丽隐杆线虫为模式生物,系统性地探讨了组织蛋白酶家族在屏障衰老过程 中的作用与机制。通过基因组数据库筛选和功能注释分析, 我们鉴定了 40 种组织蛋白酶家 族的同源基因, 并分析了它们在秀丽隐杆线虫不同组织中的表达模式。实验结果表明,这 些基因在神经、咽部、肠道和体壁肌等组织中呈现差异性表达,且部分基因在特定组织中 表现出显著的高表达,暗示了其在组织屏障功能维持中的重要作用。

为了验证这些基因在衰老相关屏障功能退变中的作用, 我们设计了一系列 RNAi 实验, 通过特异性敲低目标基因的表达, 观察其对线虫肠道和表皮屏障完整性的影响。 Smurf As-    say 实验结果显示,敲低某些组织蛋白酶同源基因后,线虫肠道屏障的通透性显著降低, 表  明这些基因在肠道屏障功能的破坏中发挥了重要作用。荧光共聚焦显微镜观察和分析进一步 表明,当敲低这些基因后,线虫表皮屏障的 HMR-1 蛋白分布更加连续且密集,空隙率显著 降低,这些该现象也说明了组织蛋白酶同源基因在表皮屏障功能的破坏中同样具有重要作  用。综合实验结果, 我们筛选出了 7 个关键基因:asp-1、cpl-1、cpr-1、cpr-3、cpr-4、cpr-6    和 ctsa-1.1,本研究不仅验证了既往关于 CPR-6 蛋白的研究结论,还扩展了组织蛋白酶家  族在衰老过程中的功能谱系。通过揭示组织蛋白酶家族成员在维持组织稳态和应对环境压  力中的作用,本研究为理解衰老相关病理变化提供了新的视角。鉴于线虫 HMR-1 蛋白与  人类连接蛋白的同源性,后续研究可将这些靶点转化到哺乳动物模型中,为临床应用提供  直接依据。此外,本研究还为延缓组织屏障功能衰退、改善老年相关疾病提供了创新疗法  的可能性,未来研究人员有望结合基因编辑技术和 RNA 干扰疗法等,实现对衰老过程的精  准干预。


\chapter{总结和期望}
本研究以秀丽隐杆线虫为模式生物,深入探讨了组织蛋白酶家族在体内屏障衰老过程  中的作用与机制。通过该实验,我们不仅验证了既往关于 CPR-6 组织蛋白酶的研究结论, 还扩展了组织蛋白酶家族在衰老过程中的功能谱系。我们成功鉴定了 40 种线虫体内组织  蛋白酶同源物,并通过 smurf assy 实验和观察HMR-1蛋白的分布筛选出了能够显著破坏肠 道和表皮屏障完整性的关键基因。这些基因的表达水平与线虫衰老过程中的屏障功能衰退  呈现显著相关性,证实了组织蛋白酶家族通过降解细胞间连接蛋白破坏组织屏障的分子机  制。这一发现为理解衰老相关病理变化提供了新的视角。

但是在进行 RNAi 实验的过程中,我们发现一些基因被敲低之后,线虫的存活率显著 下降,这可能是由于这几种基因是线虫中较为重要的基因。但至于为什么如此重要的基因 依旧会在线虫衰老的过程中破坏线虫的组织屏障,它们在线虫中的具体作用还有待研究。

本次实验为后续工作提供了多个值得深入探索的方向。比如,我们可以进一步探索这  些蛋白酶与其他衰老相关信号通路的交互作用, 这可能揭示更复杂的衰老调控网络。此外, 鉴于线虫 HMR-1 蛋白与人类连接蛋白的同源性,后续研究我们可将发现的靶点转化到哺  乳动物模型中去,为临床应用提供直接依据。

在应用层面,本研究的成果可能成为延缓组织屏障功能衰退、改善老年相关疾病的创 新疗法。同时,结合基因编辑技术和 RNA 干扰疗法,我们有望实现对衰老过程的精准干 预。

最后,实验的成功离不开实验团队成员的协作与创新。我们期待未来能将更多的基础 研究成果转化为临床应用。随着未来研究的深入,组织蛋白酶家族在衰老医学中的潜力将 被进一步挖掘,为人类健康事业开辟新的篇章。


%=================================================================
%论文后部
\backmatter


%=======%
%引入参考文献文件
%=======%
\bibdatabase{bib/database}%bib文件名称 仅修改bib/ 后部分
\printbib
% \nocite{*} %显示数据库中有的,但是正文没有引用的文献

% \Appendix

\Thanks


相聚于秋,离别在夏,我的大学生活即将落幕。总觉得来日方长,毕业遥遥可及,终于也
到我执笔于此。以为谈及这四年我会行云流水滔滔不绝,可真到下笔时,却只有百感交集。回
望漫漫求学路,并非皆是坦途,但一路前行,一路成长,无论喜悦还是悲伤,所有经历,于我
都是生命的馈赠,所有相遇,于我都是独家的珍宝。这四年来目光所及之处,皆是回忆,我度
过了人生中最青春的年华。万般不舍,心怀感激。 

涓涓恩师情,深深印于心。所谓大学者,非谓有大楼之谓也,有大师之谓也。我要向大学
期间遇到的所有课程教师以及辅导员老师表达我的感谢。任课老师的课程有趣而充实,让我在
大学期间受益匪浅,我从他们非凡的专业知识和各个领域独特的见解中获得的视野将对我未来
的生活和事业具有永恒的意义。辅导员老师的悉心照顾让我在学习上没有了后顾之忧。从知识
到校园生活,让我从一张小白纸逐渐变成了写满知识的书本。 

家人之爱,永记于心。感谢父母二十多年来的悉心培养,感谢父母一直在背后默默支持我
,正是因为你们的支持和付出,才能让我圆满的完成求学之路。失意时给予我鼓励,任性
时给予我宽容,难过时耐心听我吐露心声,你们是我前进路上最大的底气,唯有万般努力
才能成为你们的骄傲,你们永远平安、健康、快乐是我最大的心愿。
非常感谢我的朋友们,我可爱的室友们,感恩相遇,有你们的存在使我这四年并不枯燥,感恩知己,我们一起努力,一起进步,虽然最终我们都会天各一方,但是希望你们能够前程似锦,以梦为马,不负韶华。能够陪伴在我身边的朋友们,希望你们无论之后是在哪里生活、哪里工作,我们也许很难见面,但是唯有爱意不减。

最后我要特别感谢我的指导老师沈义栋老师和金卫林老师,以及我的班主任,张文华老师。桃李不言,这三位老师有着严谨的教学态度,严密的逻辑思维,丰富的学科知识,以及负责任的工作态度,让我在学习和做人方面都受益匪浅。在整个论文的定题、修改过程中也少不了他们的细心审查。我将牢记老师的教诲,不管是对于这个课题,还是对于做人的态度,我将奋力拼搏,修改错误,超越之前的自己。

道阻且长,行则将至。最后,我想感谢我自己。你走的很慢但一直前行;你真诚待人,也被人真诚相待。希望在未来,仍对世界保持好奇心,满怀期待地去热爱生活,砖一瓦不断建立自己的内心秩序,寻找自治的生活方式。我将带着春天的印记远航,尽情播撒梦想的种子,去启山去躬耕,去摇撸开拓,让夏无尽,让秋丰盈,让此生硕硕

\Grade

\end{document}
